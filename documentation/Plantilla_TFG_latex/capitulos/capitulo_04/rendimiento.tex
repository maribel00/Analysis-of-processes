\subsection{Abrir un problema por primera vez (Newcomer)}

Momento exacto en el que se consigue abrir cada problema por primera vez en el servidor, normalizado para poder compararlo (normalizado porque cada año ha durado  un tiempo diferente)

\subsection{Resolver un problema por primera vez (EarlyBird)}

Momento exacto en el que se consigue resolver cada problema por primera vez, normalizado para poder compararlo (normalizado porque cada año ha durado  un tiempo diferente)

\textbf{Falta boxplot.}

Parece que, aunque los problemas están ordenados en orden creciente de dificultad, no siempre se resuelven en el mismo orden que se espera (ANOVA p=6.01e-7, KW p=1.18e-6), es decir P1 P2 P3 P4 P5 P6 P7 P8 P9. De hecho, se ha analizado este patrón y se han encontrado las siguientes variaciones en el que la más frecuente es la esperada por el profesor.

\subsection{Siguiendo el plan del profesor (Follower)}

Se incorpora una medida de similaridad Follower en [0,1] que cuantifica cómo se parece el patrón encontrado con respecto al patrón esperado. \textbf{Falta footnote.}

\textbf{Falta tabla.}