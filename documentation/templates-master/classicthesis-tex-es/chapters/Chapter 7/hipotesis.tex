\chapter{Hipótesis de estudio}
\addcontentsline{toc}{chapter}{Hipótesis de estudio}

A pesar de que el estudio descriptivo anterior muestra unos datos muy variados, casi todos ellos son homogéneos año tras año. No obstante, el objetivo de este estudio es sentar las bases para conseguir una experiencia de aprendizaje óptima para todos los grupos de alumnos. Así pues, se va a poner énfasis en detectar a aquellos grupos que estén en riesgo de obtener un peor rendimiento o peores calificaciones. La detección temprana de éstos podría permitir al profesor actuar a tiempo para mejorar su proceso de aprendizaje. Para ello, se van a proponer una serie de métricas de calidad que se definirán sobre los registros de actividad de los alumnos con el objetivo de encontrar aquella que, con mayor certeza, identifique a los alumnos que peor están progresando.

\section{Métricas de calidad y correlaciones entre ellas}

Se definirán dos grandes grupos de métricas. El primer grupo consistirá en una colección de métricas de los grupos que solamente podrán calcularse tras la finalización de la práctica. Por el contrario, las métricas del segundo grupo podrán calcularse durante la realización de la práctica y, por tanto, serán más interesantes porque podrán facilitar la detección precoz de los grupos en riesgo.

\subsection{Medidas a posteriori del resultado de la práctica}

\begin{itemize}
\item \textbf{Achiever.} Se refiere a la nota (Grade) conseguida por el alumno. Obviamente, cuanto mayor sea ésta, mejor.
\item \textbf{Performer.} Número de problemas resueltos (Goals). Trivialmente, cuantos más objetivos haya resuelto un grupo, mejor.
\item \textbf{Terminator.} Punto de finalización de toda la práctica (Makespan). Cuanto antes, mejor (para disponer de más tiempo para repasar y corregir errores). Sin embargo, no es una métrica muy relevante.
\item \textbf{Timeburner.} Es el tiempo consumido por el alumno durante las prácticas (Duration). Este es un valor trampa, pues puede significar algo positivo (el alumno ha tardado poco en resolver la práctica porque la domina), o negativo (porque no ha podido dedicarle más tiempo).
\item \textbf{Perseverant.} Número de sesiones realizadas. Ya se ha hecho un estudio de esta medida en la sección \ref{sec:activityrecorded}.
\end{itemize}

\subsection{Medidas continuas durante la práctica}

\begin{itemize}
\item \textbf{SingleThreaded.} Cuantifica como de balanceado están los grafos que describen la actividad de un grupo en la plataforma. Además, se valora negativamente que los grupos vayan saltando de problema en problema (DAG).
\item \textbf{EarlyBird.} (FirstOpen) Como los comienzos son siempre costosos, se define  la medida EarlyBird como el promedio de tiempo de la primera apertura de cada problema.
\item \textbf{MessedUp.} Número de fails consecutivos (FailRatio) hasta resolver un problema dividido entre el numero de sesiones de ese problema. La tasa de fallo así como el tiempo dedicado a un mismo problema dependerá de la dificultad del mismo.
\item \textbf{Procastinator.} Esta medida cuantifica las posposición de tareas de los grupos. Es decir, pretende identificar a aquellos alumnos que, cuando intentan resolver un problema y no lo consiguen, saltan, curiosamente, a otros problemas más complejos (los cuales, obviamente, tampoco pueden resolver) perdiendo así un tiempo precioso.
\item \textbf{Follower.} Medida que refleja si los grupos de prácticas siguen el orden esperado de las mismas.
\end{itemize}