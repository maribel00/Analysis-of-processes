\chapter*{Motivación}
\addcontentsline{toc}{chapter}{Motivación}

\section{Introducción}

La necesidad de comprender el proceso de aprendizaje y de personalizar la enseñanza para realizar una mejor adaptación a las necesidades del individuo ha motivado la \emph{Analítica de Aprendizaje} o \emph{Learning Analytics}, disciplina que consiste en la recogida de datos de un entorno de aprendizaje y el análisis de los mismos cuyo objetivo es asistir en el proceso de aprendizaje del alumnado.

Además, el uso de laboratorios virtuales y remotos en la enseñanza está en auge. Entre muchas de sus ventajas tenemos una mayor privacidad para el alumnado, accesos planficados a los mismos o soporte para reportar la actividad de los alumnos y la calificación de los mismos.

En este trabajo fin de grado se usarán datos de siete cursos académicos obtenidos en el laboratorio virtual para sistemas multiagente de la asignatura del cuarto curso académico Desarrollo Basado en Agentes del grado de Ingeniería Informática de la Universidad de Granada (España).

El laboratorio virtual diseñado para la asignatura recoge el trabajo diario de los alumnos almacenando las interacción entre los diferentes agentes y obteniendo así un extenso dataset que nos proporciona una base sólida para el uso de diversas analíticas de aprendizaje.

Así pues, se empleará un enfoque \emph{``data-driven''} o \emph{impulsado por datos}, tomando decisiones estratégicas basándose en el análisis de los datos y en la interpretación de los mismos.

\section{Motivación}

La principal motivación de este trabajo fin de grado es, precisamente, el análisis de los procesos de aprendizaje que siguen los alumnos para que el profesorado pueda asistirles mejor durante su proceso de aprendizaje y mejorar así su rendimiento académico.

Detectar grupos con dificultades para superar sus tareas de laboratorio es primordial para que el profesor pueda ayudarles y, cuanto antes, mejor. En este caso, el uso de técnicas de minería de procesos, comúnmente asociadas a indicadores de rendimiento de los alumnos, para extraer su comportamiento de un laboratorio virtual y la utilización de aprendizaje automático supervisado para identificar estos comportamientos han demostrado ser las herramientas fundamentales para identificar el comportamiento de los alumnos y prever grupos en riesgo. Así pues, en este estudio se demuestra que los indicadores de rendimiento de los alumnos pueden ser muy útiles, tanto como su comportamiento, estrictamente desde un punto de vista topológico.

\section{Objetivos}

Los objetivos principales del proyecto serán:
\begin{itemize}
\item Identificar patrones de comportamiento indicativos de la evolución de los alumnos y del progreso de su aprendizaje, detectando, en las fases más tempranas posibles, comportamientos que pudiesen ser anómalos o que pudiesen indicar problemas de aprendizaje. Es decir, se pretende relevar, mediante la utilización de técnicas de minería de procesos, las posibles estrategias de los alumnos para cumplir los distintos objetivos de la asignatura así como desvelar su forma de trabajo habitual.
\item Sugerir a estos alumnos las medidas necesarias para que recuperen un buen ritmo de aprendizaje y un progreso adecuado a lo que el profesorado de la asignatura espera.
\end{itemize}

\section{Estructura del Trabajo Fin de Grado}

Este trabajo fin de grado consta de cuatro partes, siete capítulos y otros elementos como la portada, la autorización para su ubicación en la biblioteca de la escuela, sendos resúmenes tanto en español como en inglés (con sus respectivas palabras clave), la sección de agradecimientos, los índices general, de figuras y de cuadros así como una bibliografía, un glosario de términos y un glosario de acrónimos.

A continuación se expone un breve esquema general del contenido de las partes y capítulos de este trabajo fin de grado:

\begin{itemize}
\item Parte I: Motivaciones.
\begin{itemize}
\item
\end{itemize}
\item Parte II: Estado del arte.
\begin{itemize}
\item Capítulo 1:
\item Capítulo 2:
\item Capítulo 3:
\item Capítulo 4:
\end{itemize}
\item Parte III: Análisis descriptivo.
\begin{itemize}
\item Capítulo 5:
\item Capítulo 6:
\item Capítulo 7:
\item Capítulo 8:
\end{itemize}
\item Parte IV: Planificación del proyecto.
\begin{itemize}
\item Capítulo 9:
\item Capítulo 10:
\end{itemize}
\item Parte V: Resultados obtenidos.
\begin{itemize}
\item Capítulo 11:
\item Capítulo 12:
\item Capítulo 13:
\end{itemize}
\end{itemize}