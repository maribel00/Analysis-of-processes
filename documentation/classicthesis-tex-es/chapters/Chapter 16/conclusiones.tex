\chapter{Conclusiones}\label{sec:chapterXIV}
\addcontentsline{toc}{chapter}{Conclusiones}

Este estudio ha demostrado varias hipótesis. La más importante es que los grupos en riesgo pueden detectarse desde el primer tercio del periodo de laboratorio con una gran significancia. La segunda es que esto puede hacerse simplemente observando su comportamiento desde un punto de vista topológico, algo también intuitivo porque todos sabemos lo que hacemos cuando estamos perdidos: deambular por transiciones sin sentido. Y, en tercer lugar, la detección temprana podría deberse probablemente al hecho de que las primeras etapas son cruciales para el resto, y los errores y despistes más importantes se producen muy pronto, justo al principio.

En este trabajo se ha conseguido realizar una predicción de los grupos en riesgo de obtener un peor rendimiento así como predecir en qué intervalo de notas está la calificación de cada uno de los grupos con evidencias (análisis, descarte de outliers, correlaciones).

No obstante, el principal objetivo de este trabajo fin de grado era Visión de la Universidad de Ortega y Gasset.