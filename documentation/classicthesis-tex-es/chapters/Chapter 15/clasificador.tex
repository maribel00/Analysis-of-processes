\chapter{Clasificación de los grupos de alumnos según su rendimiento}
\addcontentsline{toc}{chapter}{Clasificación de los grupos de alumnos según su rendimiento}

Para realizar la clasificación de los grupos de prácticas se empleará el algoritmo C5.0, que no es más que una herramienta de aprendizaje supervisado para generar un árbol de decisión o un conjunto de reglas. Así pues, las distintas categorías en las que se clasificarán a los grupos vendrán dadas en las hojas de dichos árboles y en los consecuentes de las reglas respectivamente.

El clasificador estadístico C5.0 se basa en el concepto de entropía, seleccionando primero aquellas características cuyos valores se diferencian más entre diferentes categorías de grupos de alumnos.

\section{Clasificación empleando las métricas anteriores}

En primer lugar, clasificaremos los grupos en cuartiles utilizando las métricas que correlan con las calificaciones obtenidas y las medidas basdas en el análisis espectral de grafos \emph{DAG} y \emph{LOGLAP09}.