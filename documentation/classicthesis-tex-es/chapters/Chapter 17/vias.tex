\chapter{Vías futuras}\label{sec:chapterXV}
\addcontentsline{toc}{chapter}{Vías futuras}

Son múltiples las diversas propuestas que surgen a partir del trabajo aquí realizado. En primer lugar, podríamos tratar de predecir las calificaciones del alumnado de una manera más precisa, reduciendo los tamaños de los intervalos que el clasificador entrenado asignará a cada uno de los grupos. Esto podría hacerse simplemente aumentando el número de clusters en el que se subdividen los grupos de prácticas, aunque podría acarrear una pérdida de precisión por parte del clasificador al reducirse las diferencias entre clusters.

Por otra parte, se podría realizar un análisis del comportamiento de un mismo grupo a lo largo del tiempo con el fin de determinar si el comportamiento del mismo varía durante el desarrollo de las prácticas.

Por último, nótese que en este estudio se ha caracterizado el comportamiento de aquellos grupos con dificultades para resolver los problemas que se le proponen pero, ¿podemos ayudarles a recuperar un buen ritmo de aprendizaje y un progreso adecuado a lo que el profesorado de la asignatura espera? Así pues, otra vía de investigación sería proponer una serie de mejoras a los grupos que actualmente se encuentran en riesgo y ver si las mismas consiguen cambiar su comportamiento y les ayudan a progresar adecuadamente.