\chapter{Etapas del proyecto: división en objetivos}\label{chapter:objetivos}
\addcontentsline{toc}{chapter}{Etapas del proyecto: división en objetivos}

El proyecto se ha realizado siguiendo la metodología \emph{Scrum} \cite{kniberg2015scrum}. Durante la fase inicial de planificación del proyecto se realizó una subdivisión del mismo en las siguientes iteraciones:
\begin{itemize}
\item Realización de un estudio multianual y segmentado por calificaciones y primeros resultados de homogeneidad de las muestras transversal por años mediante la realización de análisis ANOVA.
\item Extracción de procesos ocultos en los datasets utilizando el programa DISCO y programación y mejora del proceso de extracción. El resultado de esta fase serán una serie de grafos representando a cada uno de los grupos de prácticas considerados donde los arcos implican una relación de dependencia temporal. Estos grafos se representarán a partir de matrices de adyacencia cuyos vértices podrán representar problemas de prácticas, pares problema de prácticas y milestone alcanzado o pares problema de prácticas y estado (\texttt{FAIL} si no se ha resuelto el problema y \texttt{OK} en caso contrario).
\item Análisis de los procesos por distintas categorías: por años (resultando ser estadísticamente iguales) y por calificación final del grupo (resultando en la existencia de diferencias). Así pues, se pretenderá caracterizar el comportamiento de los diferentes grupos de prácticas a partir de una serie de medidas de complejidad calculadas sobre los grafos anteriormente extraídos.
\end{itemize}