% Plantilla para un Trabajo Fin de Grado de la Universidad de Granada,
% adaptada para el Doble Grado en Ingeniería Informática y Matemáticas.
%
%  Autor: Mario Román.
%  Licencia: GNU GPLv2.
%
% Esta plantilla es una adaptación al castellano de la plantilla
% classicthesis de André Miede, que puede obtenerse en:
%  https://ctan.org/tex-archive/macros/latex/contrib/classicthesis?lang=en
% La plantilla original se licencia en GNU GPLv2.
%
% Esta plantilla usa símbolos de la Universidad de Granada sujetos a la normativa
% de identidad visual corporativa, que puede encontrarse en:
% http://secretariageneral.ugr.es/pages/ivc/normativa
%
% La compilación se realiza con las siguientes instrucciones:
%   pdflatex --shell-escape main.tex
%   bibtex main
%   pdflatex --shell-escape main.tex
%   pdflatex --shell-escape main.tex

% Opciones del tipo de documento
\documentclass[oneside,openright,titlepage,numbers=noenddot,openany,headinclude,footinclude=true,
cleardoublepage=empty,abstractoff,BCOR=5mm,paper=a4,fontsize=12pt,main=spanish]{scrreprt}

% Paquetes de latex que se cargan al inicio. Cubren la entrada de
% texto, gráficos, código fuente y símbolos.
\usepackage[utf8]{inputenc}
\usepackage[T1]{fontenc}
\usepackage{fixltx2e}
\usepackage{graphicx} % Inclusión de imágenes.
\usepackage{grffile}  % Distintos formatos para imágenes.
\usepackage{longtable} % Tablas multipágina.
\usepackage{wrapfig} % Coloca texto alrededor de una figura.
\usepackage{rotating}
\usepackage[normalem]{ulem}
\usepackage{amsmath}
\usepackage{textcomp}
\usepackage{amssymb}
\usepackage{capt-of}
\usepackage[colorlinks=true]{hyperref}
\usepackage{tikz} % Diagramas conmutativos.
\usepackage{minted} % Código fuente.
\usepackage[T1]{fontenc}
\usepackage{natbib}
\usepackage{algorithm} % Algoritmos
\usepackage{algpseudocode}

% Plantilla classicthesis
\usepackage[beramono,eulerchapternumbers,linedheaders,parts,a5paper,dottedtoc,
manychapters,pdfspacing]{classicthesis}

% Geometría y espaciado de párrafos.
\setcounter{secnumdepth}{0}
\usepackage{enumitem}
\setitemize{noitemsep,topsep=0pt,parsep=0pt,partopsep=0pt}
\setlist[enumerate]{topsep=0pt,itemsep=-1ex,partopsep=1ex,parsep=1ex}
\usepackage[top=1in, bottom=1.5in, left=1in, right=1in]{geometry}
\setlength\itemsep{0em}
\setlength{\parindent}{0pt}
\usepackage{parskip}
\usepackage{setspace}

% Profundidad de la tabla de contenidos.
\setcounter{secnumdepth}{3}

% Usa el paquete minted para mostrar trozos de código.
% Pueden seleccionarse el lenguaje apropiado y el estilo del código.
\usepackage{minted}
\usemintedstyle{colorful}
\setminted{fontsize=\small}
\setminted[haskell]{linenos=false,fontsize=\small}
\renewcommand{\theFancyVerbLine}{\sffamily\textcolor[rgb]{0.5,0.5,1.0}{\oldstylenums{\arabic{FancyVerbLine}}}}

% Path para las imágenes
\graphicspath{{figures/}}

\usepackage[titletoc]{appendix}

% Archivos de configuración.
\input{macros}  % En macros.tex se almacenan las opciones y comandos para escribir matemáticas.
\input{classicthesis-config} % En classicthesis-config.tex se almacenan las opciones propias de la plantilla.

% Color institucional UGR
% \definecolor{ugrColor}{HTML}{ed1c3e} % Versión clara.
\definecolor{ugrColor}{HTML}{c6474b}  % Usado en el título.
\definecolor{ugrColor2}{HTML}{c6474b} % Usado en las secciones.

% Datos de portada
\usepackage{titling} % Facilita los datos de la portada
\author{María Isabel Ruiz Martínez} 
\date{\today}
\title{Asistente para \\ el descubrimiento de \\ procesos de aprendizaje ocultos \\ durante la realización de \\ prácticas de laboratorio}

% Portada
\include{titlepage}
\usepackage{wallpaper}
\usepackage[main=spanish]{babel}

\begin{document}

\ThisULCornerWallPaper{1}{ugrA4.pdf}
\maketitle

% !TeX encoding = utf8
\newpage
\null
\thispagestyle{empty}
\newpage
\null
\thispagestyle{empty}

\hfill\vfill


\textsc{Autorización}\\\bigskip

Yo, \textbf{María Isabel Ruiz Martínez}, alumna de la titulación Doble Grado en Ingeniería Informática y Matemáticas de la \textbf{Escuela Técnica Superior de Ingenierías Informática y de Telecomunicación de la Universidad de Granada}, con DNI 75576979Z, autorizo la ubicación de la siguiente copia de mi Trabajo Fin de Grado en la biblioteca del centro para que pueda ser consultada por las personas que lo deseen.

\medskip

En Granada a \today
\begin{flushleft}
Fdo: María Isabel Ruiz Martínez

\end{flushleft}

\vfill

\endinput
% !TeX encoding = utf8
\newpage
\null
\thispagestyle{empty}
\newpage
\null
\thispagestyle{empty}

\hfill\vfill


\textsc{Informe}\\\bigskip

D. \textbf{Luis Castillo Vidal}, Catedrático de Universidad del Departamento de Ciencias de la Computación e Inteligencia Artificial de la Universidad de Granada.

\medskip

\textbf{Informa:}

\medskip

Que el presente trabajo, titulado \textit{\textbf{Asistente para el descubrimiento de procesos de aprendizaje ocultos durante la realización de prácticas de laboratorio}}, ha sido realizado bajo su supervisión por \textbf{María Isabel Ruiz Martínez}, y autorizamos la defensa de dicho trabajo ante el tribunal
que corresponda.

\medskip

En Granada a \today
\begin{flushleft}
Fdo: Luis Castillo Vidal

\end{flushleft}

\vfill

\endinput

%% !TeX encoding = utf8
\newpage
\null
\thispagestyle{empty}
\newpage
\null
\thispagestyle{empty}

\hfill\vfill

\textsc{Declaración de originalidad}\\\bigskip

Dña. María Isabel Ruiz Martínez \\\medskip

Declaro explícitamente que el trabajo presentado como Trabajo de Fin de Grado (TFG), correspondiente al curso académico 2022-2023, es original, entendida esta, en el sentido de que no ha utilizado para la elaboración del trabajo fuentes sin citarlas debidamente.
\medskip

En Granada a \today
\begin{flushleft}
Fdo: María Isabel Ruiz Martínez

\end{flushleft}

\vfill

\endinput
\pdfbookmark[1]{Agradecimientos}{agradecimientos}

\chapter*{Agradecimientos}
\thispagestyle{empty}

Aquí van mis agradecimientos.

\newpage

\onehalfspacing

\chapter*{Resumen}

Aquí va mi resumen.

\small{\spacedallcaps{Palabras clave:} analítica de aprendizaje, \; estilos de aprendizaje, \; minería de procesos, \; éxito estudiantil \; teoría de grafos, \; grafo dirigido acíclico}

\newpage
\chapter*{Summary}
%\addcontentsline{toc}{chapter}{Summary} 

\begin{otherlanguage}{english}

The study of the learning process of students when they are given a series of tasks is fundamental since it could facilitate the assimilation of new knowledge and make what is being taught accessible to the students.

The main objective of this work is, precisely, to identify the behavioural patterns of those students at risk of underachieving in order to allow early intervention by teaching staff, preventing their failure in the acquisition of new knowledge.

Thus, in this project, data registered in a virtual laboratory will be used to extract, through Process Mining techniques, a series of graphs, represented by matrices, which reflect the behaviour of the students on the platform as it will be demonstrated in this study.

An important part of this thesis has been the development of my own Process Mining tool, \emph{Graph Miner}, which is responsible for translating the virtual lab logs into the matrices that represent the behaviour of the learners. As we will see, it will make up for the drawbacks of the existing Process Mining programme, \emph{Disco}. \cite{gunther2012disco}.

In addition, supervised machine learning techniques will be used not only to predict groups at risk but also to predict in which range of grades the different groups of students are with statistical evidence. In fact, such predictions can be made with a high reliability at early stages of the development of students' practical work.

Moreover, it should be noted that, in order to carry out the classifications described in the previous paragraph, both classical measures of student performance and measures of complexity based solely on the topology of the graph represented by the characteristic matrix of each group have been used to make the rankings described in the previous paragraph, the latter proving to be just as useful as the former.

Finally, it is worth mentioning that this project inspired the presentation of the article \emph{``In heaven as on earth: The performance of students is as good as it is the digraph that describes their behavior''} \cite{SIIE23} to the \emph{XXV International Symposium on Computers in Education (SIIE)}.

\small{\spacedallcaps{Keywords:} learning analytics, \; learning styles, \; process mining, \; student success, \; graph theory, \; directed acyclic graph}

\end{otherlanguage}

% Indice
\tableofcontents

\ctparttext{
  \color{black}
  \begin{center}
    Introducción, Motivación, Objetivos y Estructura.
  \end{center}
}
\part{Motivaciones}

\documentclass[10pt,a4paper]{article}
\usepackage[utf8]{inputenc}
\usepackage{amsmath}
\usepackage{amsfonts}
\usepackage{amssymb}
\usepackage{graphicx}
\usepackage[hidelinks]{hyperref} 
\usepackage{color}
\usepackage{xcolor}
\usepackage{caption}
\usepackage{subcaption}
\author{María Isabel Ruiz Martínez}
\title{Motivación}

%Ruta absoluta en formato tipo Unix (Linux, OsX)
\graphicspath{ {/home/maribel/Escritorio/5º DGIIM/TFG/Analysis-of-processes/documentation/images} }

\begin{document}

\maketitle

La necesidad de comprender el proceso de aprendizaje y de personalizar la enseñanza para realizar una mejor adaptación a las necesidades del individuo ha motivado la \emph{Analítica de Aprendizaje} o \emph{Learning Analytics}, disciplina que consiste en la recogida de datos de un entorno de aprendizaje y el análisis de los mismos cuyo objetivo es asistir en el proceso de aprendizaje del alumnado.

\end{document}

\ctparttext{\color{black}\begin{center}
Planteamiento del problema y Minería de Procesos.
\end{center}}
\part{Estado del arte}

\chapter{Planteamiento del problema}\label{sec:chapterI}
\addcontentsline{toc}{chapter}{Planteamiento del problema}

En la asignatura Desarrollo Basado en Agentes los alumnos, organizados en grupos de 4 o 5 alumnos, se conectan a un Laboratorio remoto de la UGR que está siempre disponible para los mismos. La arquitectura del servidor remoto puede apreciarse en la Figura \ref{fig:architecture}.

\begin{figure}[H]
    \centering
    \includegraphics[width=0.60\textwidth]{estado/LARVA2122Architecturec.png}
    \caption{Arquitectura del Servidor Remoto. Por un lado, contiene el laboratorio virtual para sistemas multiagente distribuidos. Además, los alumnos también pueden consultar su progreso y el de sus compañeros a través de un Bot de Telegram. Por otro lado, el profesor también puede conocer el número de objetivos conseguidos por cada uno de sus grupos de alumnos.}
    \label{fig:architecture}
\end{figure}

Este servidor contiene varios mundos virtuales y se encarga de registrar y almacenar las interacciones con él \cite{Vidal_2016}. Cada mundo virtual es una matriz cuadrada que representa espacios abiertos (en color blanco), obstáculos (en negro) y objetivos (en rojo) tal y como se muestra en la Figura \ref{fig:map}. Los agentes de los alumnos deben entrar en uno de esos mundos virtuales, percibir su vecindario, navegar a través de los espacios abiertos (empleando alguna clase de heurística exploratoria), evitar obstáculos y tratar de llegar al objetivo. En total, cada uno de los problemas planteados requieren de cinco pasos (o \emph{milestones}) hasta su consecución.

\begin{figure}[H]
    \centering
    \includegraphics[width=0.60\textwidth]{estado/virtualworlds.png}
    \caption{Alguno de los mapas que el alumnado debe resolver. Los agentes de los grupos de estudiantes deben acceder a uno de esos mundos y deben alcanzar los objetivos (coloreados en rojo) navengado a través del mundo y evitando los obstáculos (coloreados en negro). Alguno de los mundos no son resolubles porque el objetivo no se puede alcanzar con el objetivo de forzar a los agentes de los alumnos a razonar acerca de la irresolubilidad. Las posibles trayectorias están marcadas en verde.}
    \label{fig:map}
\end{figure}

La percepción del agente de su entorno es crítica para resolver estos mundos. En este laboratorio virtual los alumnos pueden configurar cuál de los siguientes sensores estarán enchufados en sus agentes (cualquier combinación de ellos):

\begin{itemize}
	\item Un \textbf{GPS} que indica al agente sus coordenadas $(x,y)$ en el mundo virtual.
	\item Un \textbf{sensor de batería}. Cada agente está alimentado con una batería cuya capacidad es limitada y cuya carga decrece conforme el agente realiza algún movimiento. La batería nunca debe ser vaciada por completo.
	\item Un \textbf{sensor radar} que informa al agente acerca de los tipos de celdas que lo rodean con una percepción local de 5x5 (observar Figura \ref{fig:sensorb}).
	\item Un \textbf{sensor escáner} que actúa como \emph{detector del objetivo} e indica al agente la distancia al mismo medida desde cada una las celdas de su entorno 5x5 (observar Figura \ref{fig:sensorc}).
\end{itemize}

\begin{figure}[H]
\centering
\begin{subfloat}[] {
\centering
\begin{tikzpicture}[scale=0.8]
	\fill (0,0) rectangle ++ (1,1); 
    \fill (0,1) rectangle ++ (1,1);
    \fill (0,2) rectangle ++ (1,1);
    \fill (0,3) rectangle ++ (1,1);
    \fill (0,4) rectangle ++ (1,1); 
    \fill (1,4) rectangle ++ (1,1);
    \fill (2,4) rectangle ++ (1,1);
    \fill (3,4) rectangle ++ (1,1);
    \fill (4,4) rectangle ++ (1,1);
    \fill (4,3) rectangle ++ (1,1);
    \fill (4,2) rectangle ++ (1,1);
    \fill (3,2) rectangle ++ (1,1);
    \fill (2,1) rectangle ++ (1,1);
    \fill (3,1) rectangle ++ (1,1);
    \fill[red] (4,1) rectangle ++ (1,1);
 	\fill[green] (2,2) rectangle ++ (1,1); 
 	\draw[draw=gray] (0,0) grid (5,5);
\end{tikzpicture}
\label{fig:sensora}   
}
\end{subfloat}   
\hfill  
\begin{subfloat}[] {
\centering
\begin{tikzpicture}[scale=0.8]
\node at (0.5,0.5) {1};
\node at (0.5,1.5) {1};
\node at (0.5,2.5) {1};
\node at (0.5,3.5) {1};
\node at (0.5,4.5) {1};
\node at (1.5,0.5) {0};
\node at (1.5,1.5) {0};
\node at (1.5,2.5) {0};
\node at (1.5,3.5) {0};
\node at (1.5,4.5) {1};
\node at (2.5,0.5) {0};
\node at (2.5,1.5) {1};
\fill[green] (2,2) rectangle ++ (1,1);
\node at (2.5,2.5) {0};
\node at (2.5,3.5) {0};
\node at (2.5,4.5) {1};
\node at (3.5,0.5) {0};
\node at (3.5,1.5) {1};
\node at (3.5,2.5) {1};
\node at (3.5,3.5) {0};
\node at (3.5,4.5) {1};
\node at (4.5,0.5) {0};
\node at (4.5,1.5) {2};
\node at (4.5,2.5) {1};
\node at (4.5,3.5) {1};
\node at (4.5,4.5) {1};
\draw (0,0) grid (5,5);
\end{tikzpicture}
\label{fig:sensorb}   
}    
\end{subfloat}
\hfill
\begin{subfloat}[] {
\centering
\begin{tikzpicture}[scale=0.8]
\node at (0.5,0.5) {4};
\node at (0.5,1.5) {4};
\node at (0.5,2.5) {4};
\node at (0.5,3.5) {4};
\node at (0.5,4.5) {4};
\node at (1.5,0.5) {3};
\node at (1.5,1.5) {3};
\node at (1.5,2.5) {3};
\node at (1.5,3.5) {3};
\node at (1.5,4.5) {3};
\node at (2.5,0.5) {2};
\node at (2.5,1.5) {2};
\fill[green] (2,2) rectangle ++ (1,1);
\node at (2.5,2.5) {2};
\node at (2.5,3.5) {2};
\node at (2.5,4.5) {3};
\node at (3.5,0.5) {1};
\node at (3.5,1.5) {1};
\node at (3.5,2.5) {1};
\node at (3.5,3.5) {2};
\node at (3.5,4.5) {3};
\node at (4.5,0.5) {1};
\node at (4.5,1.5) {0};
\node at (4.5,2.5) {1};
\node at (4.5,3.5) {2};
\node at (4.5,4.5) {3};
\draw (0,0) grid (5,5);
\end{tikzpicture}
\label{fig:sensorc}
}       
\end{subfloat}
\caption{Un agente (representado por una celda verde en el centro de cada figura) tiene una percepción local de su entorno: solamente percibe el entorno 5x5 de celdas colindantes. El Radar \ref{fig:sensorb} muestra dicho entorno 5x5 que rodea al agente e informa de si una celda está vacía (valor $0$), de si hay un obstáculo (valor $1$) o de si hay un objetivo (valor $2$). El Escáner \ref{fig:sensorc} muestra la distancia de cada una de las celdas colindantes al objetivo.}
\label{fig:sensors}
\end{figure}

Basados en su percepción del mundo virtual, cada agente decidirá ejecutar alguna de las siguientes acciones en su entorno implementando cualquier heurística o proceso de búsqueda.

\begin{itemize}
	\item LOGIN. Entrar en cualquiera de los mundos virtuales.
	\item MOVE. Mover al agente a una de las $8$ celdas adyacentes y gastar una cierta cantidad de batería. Si la celda destino es un obstáculo o el agente se queda sin batería, el agente se rompe y sale del mundo  virtual.
	\item REFUEL. El agente recarga completamente su batería. A los agentes se les permite recargar su batería tantas veces como deseen.
\end{itemize}

\section{Funcionamiento del laboratorio virtual}\label{sec:funcionamiento}

Supongamos que una persona tiene que completar una determinada tarea con nueve pasos o milestones diferentes, numerados del $1$ al $9$, de los cuales los pasos $3$, $6$ y $9$ tienen una recompensa. Esta persona podría intentar pasar por cada una de las subtareas tantas veces como considere oportuno para conseguir todas las recompensas, llevando a cabo un registro de su actividad. Por ejemplo, la Figura \ref{fig:example} muestra uno de estos registros. Ignorando el paso 1, que sólo se utiliza para marcar el inicio del registro, esta persona ha realizado $35$ pasos, repitiendo el paso $2$ siete veces, el paso $3$ seis veces, el paso $4$ dos veces, el paso $5$ seis veces, el paso $6$ cinco veces, el paso $7$ cuatro veces, el paso $8$ cuatro veces y el paso $9$ sólo una vez. Este comportamiento puede representarse con un grafo dirigido ponderado (cíclico) donde los
nodos representan los pasos dados y las aristas se ponderan con la frecuencia detectada en el registro (Figura \ref{fig:example}), de modo que el número de veces que se ejecuta un paso viene dado por la suma de los pesos de sus aristas entrantes, lo cual denominaremos factor de grado de entrada (\emph{in-degree}) de aquí en adelante.

\begin{figure}[H]
    \centering
    \includegraphics[width=0.60\textwidth]{estado/example.png}
    \caption{La secuencia correspondiente al grafo es: $1$ $2$ $(3)$ $2$ $4$ $5$ $5$ $(6)$ $(3)$ $(3)$ $4$ $(6)$ $7$ $5$ $5$ $5$ $8$ $2$ $(6)$ $(3)$ $2$ $7$ $5$ $(6)$ $8$ $2$ $(3)$ $(3)$ $(6)$ $2$ $7$ $8$ $2$ $7$ $8$ $(9)$. Así pues, el gráfico muestra el comportamiento básico de un determinado grupo en el que se repiten $9$ pasos, tres de los cuales, representados entre paréntesis, tienen una recompensa.}
    \label{fig:example}
\end{figure}

Este registro lineal de actividades refleja exactamente cómo se almacena la realización de tareas de laboratorio en LARVA \cite{Vidal_2022}, \cite{Vidal_2023}, un laboratorio virtual \cite{Vidal_2016} diseñado con un único propósito: permitir a los estudiantes alcanzar su mayor éxito. Para ello, LARVA mantiene un registro completo de la actividad de los alumnos, por lo que su evaluación se basa no sólo en los objetivos alcanzados, sino también en su progreso. Además, cuenta con un sistema de retroalimentación multimodal \cite{Vidal_2022} para mantener a los estudiantes informados sobre su progreso, en tiempo real, gracias a mensajes de chat de Telegram enviados directamente a sus teléfonos móviles. Puede demostrarse que cuanto antes reciban el feedback sobre su actividad y cuanto más rica sea esta retroalimentación, mayor será la autorregulación y la eficacia de la experiencia de aprendizaje \cite{Keller_1968}.

Revelar este comportamiento también puede ser útil para el profesor para detectar, cuanto antes, posibles dificultades de los alumnos para completar sus tareas y permitir una intervención clave por parte del profesor. Pero, ¿cómo detectar estas desviaciones del rendimiento esperado y con qué antelación podrían detectarse? Además, ¿sería posible ignorar cualquier detalle sobre los indicadores de progreso habituales asociados al rendimiento, como el momento de los éxitos y fracasos, la perseverancia, etc., para no depender demasiado en la ``eficiencia clásica'', y centrarse en las características topológicas del comportamiento de los alumnos? Para principiantes, en la Figura \ref{fig:extreme} se muestran dos comportamientos extremos. Por un lado, en la Figura \ref{fig:bad}, después de $372$ sesiones, la mayoría de ellas quemadas en los $3$-$4$ preliminares pasos, acaba con muy pocas sesiones en los últimos problemas, una especie de piloto automático, y resuelve $7$ de cada $9$ problemas. Por otro lado, en la Figura \ref{fig:good}, después de $297$ sesiones (¿podría considerarse como un menor esfuerzo?) muestra una exploración bastante exhaustiva de las alternativas y termina con 9 de 9 problemas resueltos. Este
documento responde con éxito a todas estas preguntas a partir de un sólido análisis basado en evidencias de los registros de los últimos siete años de este laboratorio virtual. Las siguientes secciones están dedicadas a discutir trabajos similares en la literatura, a presentar el escenario y la hipótesis principal y, a continuación, a extraer las principales conclusiones tras un análisis exhaustivo de los datos registrados. Todos los conjuntos de datos mencionados en este documento y todos los artefactos de software, completamente escritos en R, están abiertos y disponibles en GitHub \footnote{\href{https://github.com/maribel00/Analysis-of-processes}{https://github.com/maribel00/Analysis-of-processes}}.

\begin{figure}[H]
\centering
\subfloat[Grafo acíclico dirigido que captura el comportamiento de un grupo con $372$ sesiones de trabajo.]{\label{fig:bad}\includegraphics[width=\textwidth]{estado/DBA_1617_P2_GJ_9.png}}\\
\subfloat[Grafo acíclico dirigido que captura el comportamiento de un grupo con $297$ sesiones de trabajo.]{\label{fig:good}\includegraphics[width=0.6\textwidth]{estado/DBA_1516_P2_GC_9.png}}
\caption{Dos comportamientos reales diferentes de grupos enfrentándose a las mismas tareas de laboratorio.}
\label{fig:extreme}
\end{figure}

\section{Descripción del escenario inicial}\label{sec:initial}

Como se ha mencionado anteriormente, este estudio se ha llevado a cabo a partir de los datos de una asignatura obligataria de 4º curso de la rama Ingeniería del Software en la Universidad de Granada registrados a través de un laboratorio virtual \cite{Vidal_2016}. Esta asignatura sigue la estructura típica del Sistema Personalizado de Instrucción \cite{Keller_1968} donde el progreso de los estudiantes se monitorea continuamente gracias a un sistema de hitos y logros continuos, proporcionando tanta retroalimentación como sea posible y tan pronto como sea posible. Así, durante el curso, los alumnos se organizan en grupos (de $4$-$5$ miembros cada uno) y cada grupo debe resolver una serie de tareas o problemas. De ahora en adelante denotaremos al conjunto de problemas por $P = \left\lbrace p_i \right\rbrace$ y al número de problemas a resolver por $n = |P|$.

Por lo general, el número de problemas es $n = 9$ y podrían considerarse como un conjunto de misiones a realizar en escenarios similares. Adicionalmente, cada problema ha sido cuidadosamente elaborado por el profesor para que requiera un nivel de dificultad creciente por parte de los estudiantes de tal forma que la estrategia para resolver $p_i$ puede no ser lo suficientemente buena resolver $p_{i+1}$ pero la estrategia para resolver $p_{i+1}$ también debe ser válida en $p_i$. Estos problemas están abiertos durante un determinado periodo de tiempo durante el cual los estudiantes pueden abrir los problemas tantas veces como necesiten, ya sea para intentar resolver el problema, para mejorar sus soluciones en estos problemas o para probar nuevas estrategias. Además de esto, cada problema $p_i$ se ha diseñado como una secuencia de $5$ hitos consecutivos, también en nivel creciente de dificultad:

\begin{equation}
p_i = \left\lbrace p_i^1 , p_i^2 , p_i^3 , p_i^4 , p_i^5 \right\rbrace
\end{equation}

siendo $p_i^1$ el suceso de sólo abrir el problema $i$ y $p_i^5$ la obtención de una solución válida para $p_i$. Estos hitos deben ser alcanzados progresivamente por los alumnos, por lo que todo acaba por empujando a los alumnos un poco hacia adelante en sus capacidades \cite{Keller_1968}. Por lo tanto, a medida que los estudiantes progresan en el laboratorio, dejan un registro de sus logros, que se llamará \emph{comportamiento}, $B^g$, de un determinado grupo $g$, y está compuesto por una secuencia de sesiones.

\begin{equation}
B^g = \left\lbrace ^sp_i^k,\dots \right\rbrace
\end{equation}

Cada \emph{sesión} del laboratorio virtual $^sp_i^k$ se etiqueta como el mayor hito $k$ alcanzado en cada problema $p_i$ y un número natural secuencial $s$ que es un registro del tiempo actual del sistema. Además, se utiliza una marca $A$ para señalar el inicio del de laboratorio y un indicador $Z$ para señalar el cierre del mismo.

\begin{equation}
B^g = \left\lbrace ^0A \right\rbrace \cup \left\lbrace ^sp_i^k,\dots \right\rbrace \cup \left\lbrace Z \right\rbrace
\end{equation}

\begin{table}[H]
\centering
\caption{El comportamiento de un grupo ficticio con $20$ sesiones de trabajo en el laboratorio virtual.}
\label{tab:sequence}
\begin{tabular}{cccccccccccc}
$^0A$ & $^1p_1^1$ & $^2p_1^3$ & $^3p_1^5$ & $^4p_1^5$ & $^5p_1^5$ & $^6p_1^4$ & $^7p_2^3$ & $^8p_2^4$ & $^9p_1^5$ & $^{10}p_1^5$ \\ \hline
$^{11}p_2^5$  & $^{12}p_1^1$ & $^{13}p_2^5$ & $^{14}p_3^2$ & $^{15}p_3^4$ & $^{16}p_1^2$ & $^{17}p_3^3$ & $^{18}p_2^3$ & $^{19}p_3^4$ & $^{20}p_3^5$ & $Z$
\end{tabular}
\end{table}

Por ejemplo, en un escenario con tres problemas, la secuencia mostrada en la Tabla \ref{tab:sequence} podría ser un comportamiento factible $B$, que ya se introdujo en la Sección \ref{sec:funcionamiento}, y ahora se explica bajo una nueva perspectiva:

\emph{``Los estudiantes se han conectado $20$ veces al laboratorio virtual. En la primera sesión se abre el problema $p_1$ sin más éxito y en la segunda sesión se deja el problema $p_1$ a medias. No obstante, en la tercera sesión se resuelve completamente $p_1$ . Los estudiantes ponen en práctica nuevas estrategias y resuelven $p_1$ dos veces más. A continuación, introducen un nuevo cambio que casi resuelve $p_1$, acaba fallando y deja $p_2$ justo en la mitad. Después, tras algunos cambios, resuelven de nuevo $p_1$ en la sesión $9$ y $p_2$ en la sesión $11$. Después, cambiaron algo en la implementación que resultó ser un fracaso en $p_1$ y acaban teniendo éxito en $p_3$ justo al final''.}

La Figura \ref{fig:groupsperyear} y Tablas \ref{tab:days} y \ref{tab:records} muestran información descriptiva de los datos registrados cada año sobre el comportamiento de $77$ grupos (en total casi $400$ alumnos). Como se puede ver, hay registros de $7$ años consecutivos y un total de $30672$ sesiones de trabajo en el laboratorio virtual. A continuación, el reto es cómo extraer un gráfico como los que se muestran en la Sección \ref{sec:funcionamiento} y cómo utilizarlos para detectar, lo antes posible, los grupos con dificultades para progresar.
\chapter{Minería de Procesos}\label{sec:chapterIII}
\addcontentsline{toc}{chapter}{Minería de Procesos}

Como se ha comentado antes, cualquier posible evaluación de los alumnos basada meramente en los objetivos alcanzados es sólo una visión estática, sin muchas pistas sobre la dinámica que les ha llevado hasta ese punto, no sabemos muy bien cómo han llegado los alumnos hasta ahí y esta información es la clave para identificar a los grupos con dificultades que podrían necesitar la ayuda del profesor. Por lo tanto, vamos a emplear técnicas de Minería de Procesos \cite{aalst2016} para extraer las diferentes secuencias de eventos correlacionados del conjunto de datos original.

\section{Extracción de los procesos con DISCO}

Para la extracción de los procesos ocultos se empleará la muy conocida suite de minería de procesos Disco \cite{gunther2012disco}. Disco es una herramienta que permite crear mapas visuales a partir de los registros en cuestión de minutos.

\emph{Fuzzy Miner} \cite{gunther2007fuzzy} es un algoritmo de minería de procesos muy flexible y sólido que está detrás de Disco. En un proceso típico minado con Disco, cada actividad del proceso se etiqueta como en la Tabla \ref{tab:sequence} e incluye las frecuencias tanto de las actividades como de las transiciones entre actividades. Por lo tanto, la frecuencia de cada actividad representa el número de veces que esta actividad aparece en $B^g$. Por otro lado, la frecuencia de cada transición entre las actividades $x$ e $y$ representa el número de veces que la actividad $x$ aparece inmediatamente antes de la actividad $y$ en $B^g$.

Para crear los diagramas de Disco, se extraerán los campos de información más importantes del dataset:
\begin{enumerate}
\item El identificador del caso, extraído de una clave aleatoria generada al principio de cada operación \texttt{LOGIN} y que distingue de manera unívoca cada sesión de trabajo de los estudiantes.
\item El agente, que se refiere al nombre del grupo de estudiantes.
\item La fecha y hora a la que se registró la transacción.
\item El campo actividad (\texttt{Activity}), que refleja la acción de los alumnos en el mundo virtual.
\item Varios campos de tipo recurso (\texttt{Resource}) que proporcionan información adicional que puede sernos de utilidad a la hora de filtrar los registros.
\end{enumerate}

Así pues, se importarán el dataset de dos maneras diferentes, con el objetivo de estudiar tanto la frecuencia con que cada problema o mapa ha sido visitado como las acciones compuestas mapa-porcentaje superado. En la primera importación la \texttt{Activity} es el problema y el identificador del caso es el grupo. En la segunda importancia, por el contrario, la \texttt{Activity} es la composición del problema y el milestione y el identificador del caso son las sesiones. Las Figuras \ref{fig:problems} y \ref{fig:compound} muestran los diagramas obtenidos en cada uno de los casos.

\begin{figure}[H]
    \centering
    \includegraphics[width=\textwidth]{problems.png}
    \caption{Análisis de procesos del dataset (\texttt{Activity} problema y \texttt{CaseId} grupo). Contiene el $100\%$ de las actividades y el $80\%$ de los caminos.}
    \label{fig:problems}
\end{figure}

\begin{figure}[H]
    \centering
    \includegraphics[width=\textwidth]{compound.png}
    \caption{Análisis de procesos del dataset (\texttt{Activity} problema-milestone y \texttt{CaseId} sesión). Contiene el $20\%$ de las actividades y el $20\%$ de los caminos.}
    \label{fig:compound}
\end{figure}

No obstante, a pesar de que tener una visión global del comportamiento de todos los grupos puede ayudar, nuestro objetivo final es poder caracterizar los comportamientos de los grupos y poder discernir, usando los datos del diagrama, si. Es por esto que nos será más interesante segmentar por grupos. Así pues, filtrando por el grupo \texttt{DBA 1516 P2 GA}, obtenemos los diagramas de las Figuras \ref{fig:problemsDBA1516P2GA} y \ref{fig:compoundDBA1516P2GA}.

\begin{figure}[H]
    \centering
    \includegraphics[width=\textwidth]{DBA1516P2GAProblems.png}
    \caption{Análisis de procesos del grupo \texttt{DBA 1516 P2 GA} (\texttt{Activity} problema y \texttt{CaseId} grupo). Contiene el $100\%$ de las actividades y el $100\%$ de los caminos.}
    \label{fig:problemsDBA1516P2GA}
\end{figure}

\begin{figure}[H]
    \centering
    \includegraphics[width=\textwidth]{DBA1516P2GACompound.png}
    \caption{Análisis de procesos del grupo \texttt{DBA 1516 P2 GA} (\texttt{Activity} problema-milestone y \texttt{CaseId} sesión). Contiene el $60\%$ de las actividades y el $80\%$ de los caminos.}
    \label{fig:compoundDBA1516P2GA}
\end{figure}

Aunque Disco tiene mucho éxito y se pueden personalizar métricas e interpretaciones del conjunto de datos fuente, no se ajusta completamente a nuestros requerimientos para desvelar la estructura del comportamiento. Tras una larga experimentación con el programa Disco, se empiezan a ver sus limitaciones. En primer lugar, aunque Disco permite el filtrado de datos, si se quiere segmentar por grupos y extraer los procesos ocultos de cada uno de los grupos, hay que seleccionar el correspondiente grupo en el filtro, extraer los diagramas correspondientes e ir cambiandolo manualmente. Dado que tenemos un total de $77$ grupos de alumnos en los siete cursos académicos que forman parte del estudio (que pueden consultarse en las Tablas \ref{tab:groups1} y \ref{tab:groups2}), es inviable seguir usando el programa.

Así pues, en este trabajo fin de grado se ha implementado nuestra propia versión del programa, la herramienta de minería de procesos \emph{GraphMiner} en R, personalizada y adaptada a las necesidades del problema, basándose principalmente en la estructura principal de FuzzyMiner \cite{gunther2007fuzzy} con algunas adiciones menores.
\chapter{Implementación de la herramienta de Minería de Procesos \emph{Graph Miner}}\label{sec:chapterIV}
\addcontentsline{toc}{chapter}{Implementación de la herramienta de Minería de Procesos \emph{Graph Miner}}

\section{Uso de grafos como representación de los grupos}

Para dar una estructura de orden mínimo al comportamiento $B^g$, suponemos que es cierto lo siguiente: el resultado de la sesión $i$ depende en su mayor parte de lo ocurrido en la sesión $i-1$ o, como mucho, de cualquiera de las sesiones inmediatamente anteriores. Si aceptamos esta suposición, entonces $B^g$ tiene una estructura de orden parcial en la que cada sesión aparece conectada a la anterior.

Además, podríamos etiquetar estas relaciones con un número natural que indica el número de veces que esta relación se ha dado en $B^g$. De este modo se, obtiene el gráfico de la Figura \textbf{referencia}, un grafo totalmente conectado \textbf{revisar} que describe el camino seguido por los estudiantes con un sentido claro de las transiciones de un estado a otro. Esta Figura también cumple el factor de grado de entrada mencionado al principio: la suma de las aristas entrantes a un nodo coincide con el número de veces que el nodo aparece en el registro. No obstante, como este estudio se centra en el éxito y el fracaso de los alumnos, vamos a colapsar este grafo. Para simplificar las cosas sólo distinguiremos dos tipos de sesiones: las que fracasan, es decir, las que alcanzan los hitos $1$ a $4$, y las que tienen éxito, es decir, las que alcanzan el hito final $5$. Así pues, notaremos a las sesiones fallidas por $^sp_i^f$ y a las sesiones exitosas por $^sp_i^s$, como en la Figura \ref{fig:examples}, que continúa el ejemplo presentado en la Sección \ref{sec:initial}. Además, la matriz de adyacencia del segundo grafo puede verse en la Ecuación \ref{eq:adyacencia}.

\begin{figure}[H]
\centering
\subfloat[Grafo cíclico dirigido que captura las relaciones de precedencia entre las sesiones de la Tabla \ref{tab:sequence} ($16$ sesiones) obtenido colapsando los vértices de la misma en exitosas o fallidas. Los nodos con doble círculo representan sesiones en las que se ha resuelto un problema.]{\label{fig:examplecycles}\includegraphics[width=0.47\textwidth]{implementación/examplecycles.png}}\qquad
\subfloat[El mismo grafo que el de la Figura \ref{fig:examplecycles} pero eliminando los ciclos
y manteniendo el mismo grado de entrada de cada vértice.]{\label{fig:examplewithoutcycles}\includegraphics[width=0.47\textwidth]{implementación/examplewithoutcycles.png}}
\caption{Grafos resultantes de la continuación del ejemplo de la Sección \ref{sec:initial}.}
\label{fig:examples}
\end{figure}

\begin{equation}\label{eq:adyacencia}
\mathcal{A}^g = 
\left(
\begin{array}{c|cccccccc}
    & A & p_1^f & p_1^s & p_2^f & p_2^s & p_3^f & p_3^s & Z \\
  \hline
  A & \textbf{0} & 5 & 0 & 0 & 0 & 0 & 0 & 0 \\
  p_1^f & 0 & \textbf{0} & 4 & 3 & 1 & 2 & 0 & 0 \\
  p_1^s & 0 & 0 & \textbf{0} & 0 & 1 & 0 & 0 & 0 \\
  p_2^f & 0 & 0 & 1 & \textbf{0} & 0 & 1 & 0 & 0 \\
  p_2^s & 0 & 0 & 0 & 0 & \textbf{0} & 1 & 0 & 0 \\
  p_3^f & 0 & 0 & 0 & 0 & 0 & \textbf{0} & 1 & 0\\
  p_3^s & 0 & 0 & 0 & 0 & 0 & 0 & \textbf{0} & 1 \\
  Z & 0 & 0 & 0 & 0 & 0 & 0 & 0 & \textbf{0}
\end{array}
\right)
\end{equation}

Esta perspectiva nos permite ver $B^g$ como un grafo dirigido $G = (V^g,E^g)$ que capta la dinámica de un grupo mientras se esfuerzan por resolver los distintos problemas que se les han planteado. Por lo tanto, el conjunto de eventos (sesiones) que los estudiantes han realizado es el conjunto de vértices del grafo (\textbf{duda}):

\begin{equation}
V^g = \left\lbrace ^0A \right\rbrace \cup \left\lbrace ^sp_i^k \in B^g \right\rbrace
\end{equation}

y cada par de sesiones consecutivas se considera una arista del grafo:

\begin{equation}
E^g = \left\lbrace <x,y>, x = {}^sp_i^r, y = {}^tp_j^l, t = s+1, x,y \in B^g\right\rbrace
\end{equation}

que, de hecho, puede representarse como su matriz de adyacencia ponderada y que nos permite representar el invariante grado de entrada.

\section{La matriz característica de un grupo}

Este grafo sigue siendo un grafo cíclico (Definición \ref{def:cyclic}) y, como estamos interesados en la representación más esencial del comportamiento de los estudiantes, también eliminamos esos ciclos quitando una cantidad mínima de aristas y conservando la invariante del grado de entrada, obteniendo el grafo esencial de la Figura \ref{fig:examplewithoutcycles}, con su respectiva matriz de distancias (Ecuación \ref{eq:adyacencia}), que denominamos matriz característica (Definición \ref{def:adjacency}) del grupo $g$, $\mathcal{A}^g$. Esta matriz característica es una representación minimal de $B^g$ y a partir de ella se pueden extraer numerosas métricas, siendo la más importante una medida de la complejidad inherente a $B^g$.

Por lo tanto, $\mathcal{A}^g[x,y] = d > 0$ significaría que $x$ precede a $y$ $d$ veces en $B^g$. Así pues, cuanto mayor sea $d$, mayor será la influencia de $x$ sobre $y$ en términos del comportamiento codificado en $\mathcal{A}^g$. En aras de la simplicidad, además del uso de valores enteros para obtener una fila o columna, también podrían utilizarse sus nombres como, por ejemplo, se muestra en la Ecuación \ref{eq:example}.

\begin{equation}\label{eq:example}
\mathcal{A}[1,2] = \mathcal{A}[p_2^f,p_3^f] = 4
\end{equation}

Podría decirse que $\mathcal{A}^g$ contiene lo que podría haber sido la \emph{experiencia} de resolver todos los problemas, es decir, una especie de huella que codifica la relación entre fracaso y éxito y la fuerza de estas relaciones. Vale la pena decir que en todo el registro de la Tabla I todas estas matrices son diferentes entre sí. Se derivarán otras dos matrices de $\mathcal{A}^g$. En primer lugar, obtendremos la matriz de adyacencia característica $\mathcal{A}'[r, c]$ que es $1$ sólo cuando $\mathcal{A}[r, c] > 0$. En segundo lugar, calcularemos la matriz de distancia mínima que se ha obtenido aplicando el camino más corto de Dijkstra, de modo que $\hat{\mathcal{A}}[r, c]$ contiene una especie de longitud de camino mínima entre ambos nodos del grafo en función del número de sesiones. Por tanto, estas estructuras son un punto de partida ideal desde el que decodificar la información sobre esta experiencia, en particular para validar si esta estructura puede estar relacionada con el éxito o no de cada grupo. Es decir, ¿un mal comportamiento de los alumnos produciría un grafo mal estructurado? Y viceversa, ¿las medidas estándar de calidad y entropía definidas puramente sobre grafos permiten detectar grupos en riesgo?
Y, la respuesta, resulta ser sí (Sección ).

\section{Resultados obtenidos}

Comparando las Figuras \ref{fig:DBA1516P2GA1} y \ref{fig:problemsDBA1516P2GA} vemos que el diagrama de la implementación propia y el original obtenido con Disco coinciden.

\begin{figure}[H]
    \centering
    \includegraphics[width=0.5\textwidth]{DBA1516P2GA1.png}
    \caption{Análisis de procesos del grupo \texttt{DBA 1516 P2 GA} (\texttt{Activity} problema y \texttt{CaseId} grupo) obtenido con la implementación propia.}
    \label{fig:DBA1516P2GA1}
\end{figure}

Además, como podemos ver en la Figura \ref{fig:DBA1516P2GA2}, hemos obtenido el mismo diagrama que en el de la Figura \ref{fig:compoundDBA1516P2GA} con la salvedad de que hemos impedido el retorno a un estado anterior (el motivo se verá más adelante). Es decir, se han eliminado ciclos.

\begin{figure}[H]
    \centering
    \includegraphics[width=\textwidth]{DBA1516P2GA2.png}
    \caption{Análisis de procesos del grupo \texttt{DBA 1516 P2 GA} (\texttt{Activity} problema-milestone y \texttt{CaseId} sesión) obtenido con la implementación propia.}
    \label{fig:DBA1516P2GA2}
\end{figure}

Además, se ha extendido la implementación de la herramienta de minería de procesos Disco \cite{gunther2012disco}, obteniendo otros diagramas que nos serán de mucha utilidad, como la agrupación en estados exitosos y fallidos (Figura) y una serie de grafos parciales considerando sólo la resolución hasta el problema $i$-ésimo, $i \in \left\lbrace 1,\dots,9 \right\rbrace$ (Figura).

A partir de ahora, estos diagramos tendrán la consideración de grafos. En particular, serán grafos dirigidos y operaremos con ellos como tales. En el siguiente capítulo se expondrán los conceptos básicos de grafos y principales resultados matemáticos que se usarán en este trabajo fin de grado.
\chapter{Teoría de grafos}
\addcontentsline{toc}{chapter}{Teoría de grafos}

En el ámbito de las matemáticas y las ciencias de la computación, se emplea el término \emph{grafo} (del griego \emph{grafos} que significa \emph{dibujo} o \emph{imagen}) para referirse a un conjunto de objetos llamados \emph{vértices} o \emph{nodos}, los cuales están unidos por enlaces conocidos como \emph{aristas} o \emph{arcos}. Estas conexiones representan las relaciones binarias que existen entre los elementos de un conjunto, y son objeto de estudio de la teoría de grafos.

\section{Grafos}

\begin{wrapfigure}{r}{0.4\textwidth}
\centering
\begin{tikzpicture}

% Posiciones fijas de los nodos
\node[circle, draw=black, fill=white] (1) at (0,0) {1};
\node[circle, draw=black, fill=white] (2) at (1,1) {2};
\node[circle, draw=black, fill=white] (3) at (1,-1) {3};
\node[circle, draw=black, fill=white] (4) at (-1,-1) {4};
\node[circle, draw=black, fill=white] (5) at (-1,1) {5};

% Arcos aleatorios
\draw[-, blue, line width=1pt] (1) to[bend left=20] (2);
\draw[-, green, line width=1pt] (2) to[bend left=20] (3);
\draw[-, red, line width=1pt] (3) to[bend left=20] (4);
\draw[-, orange, line width=1pt] (4) to[bend left=20] (5);
\draw[-, purple, line width=1pt] (5) to[bend left=20] (1);
\draw[-, violet, line width=1pt] (1) to[bend left=20] (3);

\end{tikzpicture}
\caption{Ejemplo de grafo simple.}
\label{fig:grafo1}
\end{wrapfigure}

En esta sección se introducirán las deficiones básicas que forman parte de la teoría de grafos.

\begin{definition}
Matemáticamente, un \emph{grafo} $G = (V,E)$ es una tupla de vértices $V$ y aristas $E$ que relacionan dichos vértices. Denominaremos  \emph{orden} del grafo al número de vértices del mismo ($|V|$). Por supuesto, siempre tendremos que $V \neq \emptyset$.
\end{definition}

\begin{exampleth}
El grafo dado en la Figura \ref{fig:grafo1} tiene conjunto de vértices $V=\left\lbrace 1,2,3,4,5 \right\rbrace$ y conjunto de aristas $E=\left\lbrace (1,2),(1,3),(2,3),(3,4),(4,5),(5,1) \right\rbrace$.
\end{exampleth}

\begin{definition}
Un \emph{vértice} o \emph{nodo} es la unidad fundamental de las que se componen los grafos. Los vértices en sí mismos se tratan como objetos indivisibles y sin propiedades. No obstante, pueden tener asociados una semántica dependiendo del contexto de aplicación del grafo. Por ejemplo, en el grafo \ref{fig:DBA1516P2GA2} un nodo representa la consecución de un objetivo de un problema.
\end{definition}

\begin{definition}
Una \emph{arista} representa una relación entre dos vértices de un grafo. Las aristas se denotan por $(u,v) \in E$ donde $u,v\in V$. Visualmente, se representan como las líneas que unen los vértices que forman parte de la definición de la misma.
\end{definition}

\begin{definition}
Un \emph{grafo podenderado} es un grafo cuyas aristas tienen un peso o valor asociado.

Formalmente, se puede definir como un trío ordenado $G=(V,E,W)$ donde $V=\left\lbrace v_1, \dots, v_n \right\rbrace$ es un conjunto de vértices, $E = \left\lbrace e_1, \dots, e_m \right\rbrace$ y $W = \left\lbrace w_1,\dots,w_m\right\rbrace$ es el conjunto de pesos asociados a cada arista.
\end{definition}

\deactivatequoting
\begin{figure}[H]
  \centering
\begin{minipage}[t]{0.45\linewidth}
\centering
\begin{tikzpicture}
  \node[circle, draw] (1) at (0,0) {1};
  \node[circle, draw] (2) at (2,0) {2};
  \node[circle, draw] (3) at (4,0) {3};
  \node[circle, draw] (4) at (0,2) {4};
  \node[circle, draw] (5) at (2,2) {5};
  \node[circle, draw] (6) at (4,2) {6};

  \draw[red, line width=1pt] (1) -- node[midway, above] {2} (2);
  \draw[blue, line width=1pt] (2) -- node[midway, above] {1} (3);
  \draw[green, line width=1pt] (3) -- node[midway, right] {3} (6);
  \draw[cyan, line width=1pt] (4) -- node[midway, below] {4} (5);
  \draw[orange, line width=1pt] (5) -- node[midway, below] {2} (6);
  \draw[purple, line width=1pt] (4) -- node[midway, right] {1} (1);
  \draw[magenta, line width=1pt] (2) -- node[midway, right] {4} (5);
\end{tikzpicture}
\caption{Ejemplo de grafo ponderado.}
\label{fig:grafo2}
\end{minipage}
\hspace{0.5cm}
\begin{minipage}[t]{0.45\linewidth}
\centering
\begin{tikzpicture}

% Posiciones fijas de los nodos
\node[circle, draw=black, fill=white] (1) at (0,0) {1};
\node[circle, draw=black, fill=white] (2) at (1,1) {2};
\node[circle, draw=black, fill=white] (3) at (1,-1) {3};
\node[circle, draw=black, fill=white] (4) at (-1,-1) {4};
\node[circle, draw=black, fill=white] (5) at (-1,1) {5};

% Arcos aleatorios
\draw[->, blue, line width=1pt] (1) to[bend left=20] (2);
\draw[->, green, line width=1pt] (2) to[bend left=20] (3);
\draw[->, red, line width=1pt] (3) to[bend left=20] (4);
\draw[->, orange, line width=1pt] (4) to[bend left=20] (5);
\draw[->, purple, line width=1pt] (5) to[bend left=20] (1);
\draw[->, violet, line width=1pt] (1) to[bend left=20] (3);

\end{tikzpicture}
\caption{Ejemplo de grafo dirigido.}
\label{fig:grafo3}
\end{minipage}
\end{figure}
\activatequoting

\begin{definition}
Un \emph{grafo no dirigido} es un grafo cuyas aristas representan relaciones simétricas y  carecen de sentido definido. Es decir, la arista $(u,v)$ es idéntica a la arista $(v,u)$. Es decir, las aristas no son pares ordenados sino conjuntos $\left\lbrace u,v \right\rbrace$ (o $2$-multiconjuntos) de vértices.

Un grafo no dirigido podrá tener, a lo más, $\frac{|V|^2}{2}$ aristas.
\end{definition}

\begin{definition}
Se denomina \emph{grafo dirigido} o \emph{digrafo} a aquellos grafos cuyas aristas tengan un sentido definido. En un digrafo, cada arista se representa como un par ordenado de dos vértices. Por ejemplo, $(u,v)$ denota la arista que va de $u$ hacia $v$ (desde el primer vértice hasta el segundo vértice).

Los grafos no dirigidos se pueden ver como un caso particular de los grafos dirigidos en tanto que son grafos dirigidos simétricos.

Mientras que en un grafo no dirigido se tiene que $E \subseteq \{x \in \mathcal{P}(V) : |x| = 2\}$ (es decir, $E$ es un conjunto de pares no ordenados de elementos de $V$), cuando el grafo es dirigido se tiene que $E$ es un conjunto de pares ordenados $(i,j) \in V \times V$.
\end{definition}

\begin{exampleth}
En la Figura \ref{fig:grafo3} se muestra un ejemplo de grafo dirigido mientras que en la Figura \ref{fig:grafo1} tenemos un ejemplo de grafo no dirigido.
\end{exampleth}

\begin{figure}[H]
    \centering
\begin{tikzpicture}
  \node[circle, draw] (1) at (0,0) {1};
  \node[circle, draw] (2) at (2,0) {2};
  \node[circle, draw] (3) at (4,0) {3};
  \node[circle, draw] (4) at (0,2) {4};
  \node[circle, draw] (5) at (2,2) {5};
  \node[circle, draw] (6) at (4,2) {6};

  \draw[red, line width=1pt] (1) -- (2);
  \draw[blue, line width=1pt] (2) -- (3);
  \draw[green, line width=1pt] (3) -- (4);
  \draw[yellow, line width=1pt] (4) -- (5);
  \draw[orange, line width=1pt] (5) -- (6);
  \draw[purple, line width=1pt] (6) -- (1);

  \draw[red, line width=1pt] (1) to [out=90, in=180, loop, distance=1cm] (1);
  \draw[blue, line width=1pt] (2) to [out=90, in=0, loop, distance=1cm] (2);
  \draw[green, line width=1pt] (3) to [out=90, in=0, loop, distance=1cm] (3);
  \draw[pink, line width=1pt] (5) to [out=90, in=90, loop, distance=1cm] (6);
  \draw[cyan, line width=1pt] (2) to [out=270, in=270, loop, distance=1cm] (3);
\end{tikzpicture}
	\caption{Ejemplo de multigrafo.}
	\label{fig:grafo4}
\end{figure}

\begin{definition}
Un \emph{grafo conexo} es un grafo en que todos sus vértices están conectados por un camino o por un semicamino dependiendo de si el grafo es no dirigido o dirigido.

De lo contrairo, si algún grafo no cumple la propiedad anterior se dirá que es \emph{disconexo}.
\end{definition}

\begin{definition}
Un \emph{bucle} es una arista que relacionado un vértice consigo mismo.
\end{definition}

\begin{definition}
En un grafo $G=(V,E)$, se dice que dos aristas son \emph{paralelas} o \emph{múltiples} si el vértice inicial y el vértice final de las mismas coinciden. 

Los grafos que permiten la existencia de bucles y aristas múltiples se denominan \emph{multigrafos}. Por el contrario, los grafos sin bucles y sin aristas paralelas se denominarán \emph{simples}.
\end{definition}

\begin{exampleth}
En la Figura \ref{fig:grafo2} tenemos un ejemplo de grafo ponderado.

En la Figura \ref{fig:grafo1} se muestra un ejemplo de grafo simple. Por otro lado, en la Figura \ref{fig:grafo4} podemos ver un multigrafo.
\end{exampleth}

\begin{definition}
En un grafo $G=(V,E)$ dos vértices se dirán \emph{adyacentes} (o \emph{vecinos}) si están relacionados por al menos una arista. Es decir, dos vértices $u,v \in V$ son adjacentes si $\exists e \in E$ tal que $e = (u,v)$.

La \emph{matriz de adyacencia} de un grafo es una matriz cuadrada de dimensión $|V| \times |V|$ que se utiliza como forma de representar las relaciones binarias entre los nodos del mismo. La denotaremos por $A = (a_{ij})_{1\leq i,j\leq |V|}$.

Si tenemos que $G$ es un grafo no dirigido, entonces $a_{ij} = 1$ y $a_{ji} = 1$ si el vértice $v_i$ es adyacente al vértice $v_j$ y $a_{ij} = a_{ji} = 0$ en caso contrario. Si el grafo $G$ es dirigido, entonces tendremos que $a_{ij} = 1$ si y sólo si existe $e \in E$ tal que $e = (v_i,v_j)$ y $a_{ij} = 0$ en caso contrario.

Por último, si tenemos un grafo ponderando, entonces se sustiuirá en valor de $1$ en los casos anteriores por el peso de las aristas correspondientes.
\end{definition}

\begin{exampleth}
Tenemos que la matriz de adyacencia del grafo de la Figura \ref{fig:grafo1} es:
\begin{equation}
\begin{pmatrix}
0 & 1 & 1 & 0 & 1\\
1 & 0 & 1 & 0 & 0\\
1 & 1 & 0 & 1 & 0\\
0 & 0 & 1 & 0 & 1\\
1 & 0 & 0 & 1 & 0
\end{pmatrix}
\end{equation}
\end{exampleth}

\begin{definition}
Sea $G=(V,E)$ un grafo no dirigido y sea $v \in V$ un vértice suyo. Se denomina grado del vértice $v$ al número de aristas incidentes al vértice y se denotará de ahora en adelante por $\text{deg}(v)$.

Al conjunto de todos los vértices adyacentes a un vértice dado se le denominará \emph{vecindad} del vértice en cuestión. Formalmente, la vecindad de un vértice $v \in V$ es el conjunto
\begin{equation}
N(v) = \left\lbrace u \in V | \left\lbrace v,u\right\rbrace \in E \right\rbrace
\end{equation}

Así pues, el grado de un vértice $v \in V$ puede definirse como el módulo de su vecindario: $\text{deg}(v) = |N(v)|$.

En el caso de los grafos dirigidos se distingue entre el \emph{grado de entrada} $\text{deg}^-(v)$ (número de aristas que tienen a $v$ como el vértice final) y el \emph{grado de salida} $\text{deg}^+(v)$ (número de ariastas que tienen a $v$ como vértice inicial). 
\end{definition}

% Meter Lema del apetrón de manos?

\begin{figure}[H]
\centering
\begin{tikzpicture}
  \foreach \x in {1,...,8}{
    \node[circle, draw] (\x) at ({45*(\x-1)}:2) {\x};
  }

  \foreach \x in {1,...,7}{
    \foreach \y in {\x,...,8}{
      \pgfmathsetmacro\randhue{rnd}
      \definecolor{mycolor}{hsb}{\randhue, 1, 1}
      \draw[mycolor, line width=1pt] (\x) -- (\y);
    }
  }
\end{tikzpicture}
\caption{Ejemplo de grafo completo.}
	\label{fig:grafo5}
\end{figure}

\begin{definition}
Un grafo en el que todos sus vértices tienen el mismo grado (de entrada, en el caso de los grafos dirigidos) se denomina \emph{regular}. Además, un grafo con vértices de grado $k$ se llamará $k$-regular.
\end{definition}

\begin{definition}
Un \emph{grafo completo} $G=(V,E)$ es un grafo no dirigido simple en el que para cada par de vértices $u, v\in V$ existe una arista $e \in E$ tal que $e = \left\lbrace u,v\right\rbrace$.

El \emph{grafo completo de $n$ vértices} se denotará por $K_n$. Así pues, $K_n$ tendrá $frac{n\cdot (n-1)}{2}$ aristas y es un grafo regular de grado $n-1$.
\end{definition}

\begin{figure}[H]
  \centering
\begin{minipage}[t]{0.45\linewidth}
\centering
\begin{tikzpicture}
  \foreach \x in {1,...,8}{
    \node[circle, draw] (\x) at ({45*(\x-1)}:2) {\x};
  }

  \foreach \x in {1,...,7}{
      \pgfmathsetmacro\randhue{rnd}
      \definecolor{mycolor}{hsb}{\randhue, 1, 1}
      \pgfmathtruncatemacro{\next}{\x + 1}
      \draw[mycolor, line width=1pt] (\x) -- (\next);
  }
  \pgfmathsetmacro\randhue{rnd}
  \definecolor{mycolor}{hsb}{\randhue, 1, 1}
  \draw[mycolor, line width=1pt] (8) -- (1);
\end{tikzpicture}
\caption{Ejemplo de grafo ciclo.}
	\label{fig:grafo6}
\end{minipage}
\hspace{0.5cm}
\begin{minipage}[t]{0.45\linewidth}
\centering
    \begin{tikzpicture}
  \foreach \x in {1,...,8}{
    \node[circle, draw] (\x) at ({45*(\x-1)}:2) {\x};
  }
  \node[circle, draw] (9) at (0,0) {9};

  \foreach \x in {1,...,7}{
      \pgfmathsetmacro\randhue{rnd}
      \definecolor{mycolor}{hsb}{\randhue, 1, 1}
      \pgfmathtruncatemacro{\next}{\x + 1}
      \draw[mycolor, line width=1pt] (\x) -- (\next);
      \draw[mycolor, line width=1pt] (\x) -- (9);
  }
  \pgfmathsetmacro\randhue{rnd}
  \definecolor{mycolor}{hsb}{\randhue, 1, 1}
  \draw[mycolor, line width=1pt] (8) -- (1);
  \draw[mycolor, line width=1pt] (8) -- (9);
\end{tikzpicture}
\caption{Ejemplo de grafo rueda.}
\label{fig:grafo7}
\end{minipage}
\end{figure}

\begin{definition}
Un \emph{grafo ciclo} o simplemente un \emph{ciclo} es un grafo que consiste en un camino simple cerrado. Esto es, hay un único camino en el que no se repite ningún vértice salvo el primero con el último.

Denotaremos a un grafo ciclo de $n$ vértices por $C_n$. Si consideramos que es un grafo no dirigido, cada vértice tendrá un vecindario de tamaño $2$ y, por tanto, será un grafo $2$-regular. Por el contrario, si tenemos un grafo dirigido, será un grafo $1$-regular.
\end{definition}

\begin{definition}
Un grafo rueda es un grafo de $n$ vértices (denotado usualmente por $W_n$) es un grafo que se obtiene al añadir un único vértice a un grafo ciclo de $n-1$ vértices, conectando el nuevo el vértice a todos los ya existentes. Es decir, el nuevo vértice será adyacente a todos los vértices del grafo $C_{n-1}$.
\end{definition}

\begin{exampleth}
En las Figuras \ref{fig:grafo5}, \ref{fig:grafo6} y \ref{fig:grafo7} podemos ver un grafo completo, un grado ciclo y un grafo rueda respectivamente.
\end{exampleth}

\begin{definition}
Diremos que un grafo es \emph{cíclico} si contiene al menos un grafo ciclo. Por el contrario, se dirá que un grafo es \emph{acíclico} si no contiene ningún ciclo.

No obstante, en este trabajo fin de grado nos centraremos en los llamados \emph{grafos dirigidos acíclicos} o \emph{DAG} (\emph{Directed Acyclic Graphs}, en inglés) que no son más que grafos dirigidos desprovistos de ciclos.
\end{definition}

\begin{figure}[H]
\centering
\begin{tikzpicture}

% Posiciones fijas de los nodos
\node[circle, draw=black, fill=white] (1) at (0,0) {1};
\node[circle, draw=black, fill=white] (2) at (1,1) {2};
\node[circle, draw=black, fill=white] (3) at (1,-1) {3};
\node[circle, draw=black, fill=white] (4) at (-1,-1) {4};
\node[circle, draw=black, fill=white] (5) at (-1,1) {5};

% Arcos aleatorios
\draw[->, blue, line width=1pt] (1) to[bend left=20] (2);
\draw[->, green, line width=1pt] (2) to[bend left=20] (3);
\draw[->, red, line width=1pt] (3) to[bend left=20] (4);
\draw[->, orange, line width=1pt] (4) to[bend left=20] (5);
\draw[->, violet, line width=1pt] (1) to[bend left=20] (3);

\end{tikzpicture}
\caption{Ejemplo de grafo acíclico dirigido con $5$ nodos.}
\label{fig:grafo8}
\end{figure}

\begin{exampleth}
En la Figura \ref{fig:grafo1} tenemos un grafo dirigido con ciclos o cíclico (contiene, por ejemplo, el $1$-$3$-$4$-$5$). Sin embargo, eliminado una de las aristas del mismo obtenemos el grafo de la Figura \ref{fig:grafo8}, que es acíclico. 
\end{exampleth}

\begin{definition}
Un grafo conexo acíclico no dirigido se denominará \emph{árbol}. Por otro lado, un \emph{árbol orientado} o \emph{poliárbol} será un grafo dirigido acíclico cuyo grafo no dirigido subyacente es un árbol. De otra manera, si cambiamos sus aristas dirigidas por no diridas, se obtendía un grafo no dirigido conexo y acíclico.
\end{definition}

\begin{definition}
Un \emph{árbol de expansión} de un grafo conexo no dirigido $G$ es un subgrafo suyo que es árbol y que contiene a todos sus vértices.

El \emph{número de árboles de expansión} de un grafo conexo $G$, habitualmente denotado por $t(G)$, es un invariante importante en la teoría de grafos. Éste puede obtenerse mediante el denominado \emph{Teorema de Kirchhoff}. Este teorema demuestra que el número de árboles de expansión de un grafo puede obtenerse en tiempo polinómico a partir del determinante de una submatriz de la \emph{matriz Laplaciana} del grafo. Más aún, nos dice que éste número es igual a cualquier cofactor de la matriz Laplaciana. El Teorema de Kirchhoff es una generalización de la \emph{fórmula de Cayley}, que proporciona el número de árboles de expansión en el caso de un grafo completo y que veremos a continuación.
\end{definition}

\begin{proposition}\label{prop:1}
Dado un grafo completo $K_n = (V,E)$ con $V=\left\lbrace v_1,v_2,\dots,v_n\right\rbrace$, la fórmula de Cayley establece que el número de árboles de expansión del mismo es $t(K_n) = n^{n-2}$.
\end{proposition}

En 1918, el alemán H. Prüfer obtuvo una elegante correspondencia biyectiva entre árboles etiquetados con $n$ vértices y sucesiones de longitud $n-2$, denominadas \emph{códigos de Prüfer}.

\begin{definition}
La definición del \emph{código de Prüfer} de un árbol $T = (V,E)$ no trivial, denotado por $P(T)$, es recursiva. Si $|V| = 2$ entonces $T$ consiste de una sola arista y $P(T) = \emptyset$. Supongamos ahora que el código de Prüfer de cualquier árbol con $n$ vértices está definido  y sea $T = (V = \left\lbrace v_1, v_2,\dots v_n, v_{n+1} \right\rbrace,E)$ un árbol con $n+1$ vértices. Sea
\begin{equation}
v = \min \left\lbrace i \in \left\lbrace 1,2,\dots,n,n+1 \right\rbrace | \text{ deg}(v_i) = 1\right\rbrace
\end{equation}
y sea $u$ el único vértice adyacente a $v$ en $T$. Por lo tanto, $T' = T - v$ es un árbol con $n$ vértices y $P(T-v)$ está bien definido por hipótesis de inducción. El código de Prüfer de $T$ se definirá de la siguiente forma:
\begin{equation}
P(T) = (u,P(T'))
\end{equation}
\end{definition}

\begin{figure}[H]
\centering
\begin{tikzpicture}
  \coordinate (A) at (90:2);
  \coordinate (B) at (18:2);
  \coordinate (C) at (-54:2);
  \coordinate (D) at (-126:2);
  \coordinate (E) at (162:2);

  \foreach \x/\label in {A/1, B/2, C/3, D/4, E/5}{
    \node[circle, draw] (\x) at (\x) {\label};
  }

  \draw[magenta, line width=1pt] (E) -- (B);
  \draw[blue, line width=1pt] (B) -- (C);
  \draw[cyan, line width=1pt] (D) -- (E);
  \draw[orange, line width=1pt] (E) -- (A);
\end{tikzpicture}
\caption{Ejemplo de árbol con $5$ nodos.}
\label{fig:arbol1}
\end{figure}

\begin{exampleth}
Consideremos el árbol de la Figura \ref{fig:arbol1}. Tenemos que el vértice de grado uno con la numeración más pequeña es el vértice $1$. Este vértice únicamente es adayacente al vértice número $5$. Así pues, $P(T) = (5,P(T-1))$. Cuando eliminamos el primer vértice, obtenemos que el vértice de grado uno de menor numeración es el $3$, cuyo único vértice adyacentes es el $2$, lo que conduce a que $P(T) = (5,2,P(T-\left\lbrace 1,3 \right\rbrace))$. Eliminando ahora este tercer vértice, tenemos que el vértice $2$ es el de menor numeración cuyo grado es $1$ y su único nodo adyacente es el número $5$. Esto hace que $P(T) = (5,2,5,P(T-\left\lbrace 1,3,2 \right\rbrace))$. Como $T-\left\lbrace 1,3 \right\rbrace$ es un árbol con dos vértices, $P(T-\left\lbrace 1,3 \right\rbrace) = \emptyset$ y $P(T) = (5,2,5)$.

Este ejemplo pone de manifiesto que no es necesario que todos los vértices aparezcan en el códgo de Prüfer y que pudiera ocurrir que un mismo vértice aparezca más de una vez en los mismos. De hecho, el número de veces que un vértice aparece en el código de Prüfer depende del grado de dicho vértice. Este resultado se verá en el siguiente Lema.
\end{exampleth}

\begin{proposition} % En realidad es un Lema
Sea $c_i$ el número de veces que aparece el número $i$ en el código de Prüfer de un árbol $T = (V=\left\lbrace 1,\dots, n \right\rbrace, E)$ con $n \geq 3$ vértices. Entonces $\text{deg}(i) = c_i +1$.
\end{proposition}

\begin{proof}
Si $n = 3$ entonces $P(T)$ consiste de un solo número, correspondiente al vértice de grado dos.

Supongamos ahora que el resultado es cierto para todo árbol $T = (V=\left\lbrace 1,\dots, n \right\rbrace, E)$ con $n \geq 3$ vértices. Sea $T' = (V'=\left\lbrace 1,\dots, n,n+1 \right\rbrace, E')$, $v = \min \left\lbrace i \in V' | \text{ deg}(i) = 1 \right\rbrace$ y sea $u$ el único vértice adyacente a $v$ en $T'$. Así pues, $P(T') = (u,P(T'-v))$. Para cada $i \in V'$ sea $b_i$ el número de veces que aparece $i$ en $P(T'-v)$. Por hipótesis de inducción, $\text{deg}_{T'-v}(i) = b_i + 1$.

Además, si $i \neq u$ entonces $c_i = b_i$ y $\text{deg}_{T'}(i) = \text{deg}_{T'-v}(i)$. Por lo tanto, en este primer caso, $\text{deg}_{T'}(i) = c_i + 1$. Por otra parte, si $i = u$ entonces $c_i = b_i + 1$ y $\text{deg}_{T'}(i) = \text{deg}_{T'-v}(i) +1$. Por lo tanto, $\text{deg}_{T'}(i) = b_i + 2 = c_i + 1$.
\end{proof}

El siguiente resultado muestra que el código de Prüfer define una función inyectiva del conjunto de árboles generadores con $n$ vértices al conjunto de palabras de longitud $n-2$ del alfabeto $\left\lbrace 1,2,\dots, n\right\rbrace$.

\begin{proposition}\label{prop:2}
Si $T$ y $T'$ son dos árboles con $n \geq 3$ vértices numerados tales que $P(T) = P(T')$, entonces $T = T'$.
\end{proposition}

\begin{proof}
Si $n = 3$, entonces $P(T)$ consiste de un solo número, correspondiente al único vértice de grado dos de $T$. Como $P(T) = P(T')$, este vértice también es el único vértice de grado dos de $T'$ y tenemos que $T = T'$.

Supongamos ahora que el resultado es cierto para cualesquiera dos árboles con $n$ vértices numerados. Sean ahora $T$ y $T'$ dos árboles con $n+1$ vértices numerados tales que $P(T) = P(T')$.  Sea $v$ el mínimo elemento de $\left \lbrace 1,\dots,n,n+1\right \rbrace$ que no aparece en $P(T)$. Por el lema anterior $\text{deg}_T (v) = 1$ y $\text{deg}_{T'} = 1$. Así pues, existe un único $u$ tal que $u$ es adyacente a $v$ en $T$ y existe un único $u'$ tal que $u'$ es adyacente a $v$ en $T'$. De ahí obtenemos que $P(T) = (u,P(T-v))$ y que $P(T') = (w,P(T-v'))$. Como $P(T) = P(T')$, se sigue que $u = u'$ y, por lo tanto, $P(T-v) = P(T'-v')$. Por lo que, por hipótesis de inducción, $T-v = T'-v$. Concluimos así que $T=T'$.
\end{proof}

A continuación veremos que a cada palabra de longitud $n-2$ del alfabeto $\left\lbrace 1,2,\dots,n \right\rbrace$ le corresponde un árbol cuyo código de Prüfer es esa palabra.

\begin{proposition}\label{prop:3}
Sea $n \geq 3$. Si $L = (u_1,u_2,\dots,u_{n-2})$ es una lista cuyos elementos pertenecen al conjunto $V = \left\lbrace 1,\dots,n\right\rbrace$, entonces existe un árbol con vértices numerados (o etiquetado) $T$ tal que $P(T) = L$.
\end{proposition}

\begin{proof}
Si $n=3$ entonces $L = (u_1)$ y $T$ es la trayectoria de longitud dos cuyo vértice interno es $u_1$.

Supongamos ahora que el resultado es cierto para toda lista de longitud $n-2$. Sea $L = (u_1,u_2,\dots,u_{n-1})$ una lista de longitud $n-1$. Sea $v$ el elemento más pequeño de $V$ que no aparece en $L$. Por hipótesis de inducción existe un árbol $T' = (V',E')$ con conjunto de vértices $V_{T'} = \left\lbrace 1,\dots,n+1 \right\rbrace - \left\lbrace v \right\rbrace$, tal que $P(T') = (u_2,u_3,\dots,u_{n-1})$. Sea $e$ la arista que une a $v$ con $u_1$ y sea $T = (V' + v, E' + e)$. Tenemos que $T$ es un árbol y que $P(T) = (u_1,P(T')) = L$ tal y como se quería demostrar.
\end{proof}

El siguiente ejemplo pone en práctica el procedimiento descrito en el lema anterior para construir un árbol generador cuyo código de Prüfer sea igual a una lista dada.

\begin{exampleth}
Consideremos la lista $(3,5,3,1)$. La longitud de esta lista es $4$, por lo que corresponde a un árbol con conjunto de vértices $V = \left\lbrace 1,2,3,4,5,6 \right\rbrace$. El primer vértice que no aparece en la lista es $2$, el cual deber ser adyacente al vértice $3$ (Figura \ref{fig:arbol2}). Consideremos ahora la sublista $(5,3,1)$, el primer vértice del conjunto $\left\lbrace 1,3,4,5,6 \right\rbrace$ (obtenido al eliminar el vértice $2$) que no aparece en la lista es el $4$, que debe ser adyacente al vértice $5$. Análogamente, el primer vértice del conjunto $\left\lbrace 1,3,5,6 \right\rbrace$ que no aparece en la sublista $(3,1)$ es el $5$, el cual debe ser adyacente al vértice $3$, y el primer vértice del conjunto $\left\lbrace 1,3,6 \right\rbrace$ que no aparece en la sublista $(1)$ es el $3$, el cual debe ser adyacente al vértice $1$. Finalmente, obtenemos la lista vacía y el conjunto $\left\lbrace 1,6 \right\rbrace$, lo cual indica que el vértice $1$ debe ser adyacente al vértice $6$. La Figura \ref{fig:graforesult} es el árbol asociado a la lista $(3,5,3,1)$.
\end{exampleth}

\begin{figure}[H]
\centering
\begin{tikzpicture}
  \coordinate (A) at (90:2);
  \coordinate (B) at (30:2);
  \coordinate (C) at (-30:2);
  \coordinate (D) at (-90:2);
  \coordinate (E) at (-150:2);
  \coordinate (F) at (150:2);

  \foreach \x/\label in {A/1, B/2, C/3, D/4, E/5, F/6}{
    \node[circle, draw] (\x) at (\x) {\label};
  }

  \draw[blue, line width=1pt] (B) -- (C);
  \draw[green, line width=1pt] (C) -- (E);
  \draw[magenta, line width=1pt] (D) -- (E);
  \draw[orange, line width=1pt] (C) -- (A);
  \draw[purple, line width=1pt] (F) -- (A);
\end{tikzpicture}
\caption{Ejemplo de árbol con $6$ nodos.}
\label{fig:arbol2}
\end{figure}

Finalmente, se procederá a demostrar la fórmula de Cayley \ref{prop:1}:
\begin{proof}
Si $n = 1$ o $n = 2$ el resultado es trivialmente cierto. Si $n \geq 3$ de las Proposiones \ref{prop:2} y \ref{prop:3} se sigue que existe una correspondencia biyectiva entre el conjunto de árboles generadores con $n$ vértices y el conjunto de palabras de longitud $n-2$ del alfabeto $\left\lbrace 1,2,\dots,n \right\rbrace$. Como el número de palabras de longitud $n-2$ de un alfabeto con $n$ elementos es $n^{n-2}$, entonces hay $n^{n-2}$ árboles de expansión distintos de un grafo completo con $n$ vértices.
\end{proof}

A continuación, se introducirá el concepto de matriz laplaciana de un grafo que, junto con el Teorema de Kirchhoff nos permitirá calcular el número de árboles de expansión de un grafo arbitrario.

\begin{definition}
La matriz laplaciana (también conocida como matriz de admitancia o matriz de Kirchhoff) es una representación matricial de un grafo muy utilizada en la Teoría espectral de grafos, cuyo objetivo es el estudio de las propiedades de los grafos en relación de los polinomios caracterísicos, valores y vectores propios de las matrices asociadas a los mismos.

Para un grafo simple $G$ con vértices $V = (v_1,\dots,v_n)$ los elementos de la matrix laplaciana $L_{n\times n}$ se definen como sigue:

\begin{equation}
L_{i,j} = 
\begin{cases}
    \text{deg}(v_i) & \text{si } i=j \\
    -1 & \text{si } i\neq j \text{ y }v_i \text{ es adyacente a }v_j\\
    0 & \text{en cualquier otro caso}
\end{cases}
\end{equation}

Equivalentemente, se tiene que $L = D - A$ donde $D$ es la matriz de grados del grafo (matriz diagonal cuyos elementos no nulos son los grados de cada uno de los vértices) y $A$ es la matriz de adyacencia del grafo.
\end{definition}

\begin{exampleth}
Se tiene que la matriz laplaciana del grafo simple de la Figura \ref{fig:grafo1} es la siguiente:

\begin{equation}
\begin{pmatrix}
3 & -1 & -1 & 0 & -1\\
-1 & 2 & -1 & 0 & 0\\
-1 & -1 & 3 & -1 & 0\\
0 & 0 & -1 & 2 & -1\\
-1 & 0 & 0 & -1 & 2
\end{pmatrix}
\end{equation}
\end{exampleth}

Este concepto se puede generalizar al caso de grafos ponderados, donde las matrices de adyacencia pueden contener números naturales distintos de ceros y unos. Además, también se puede generalizar a grafos dirigidos, utilizando en vez de la matriz de grados del grafo la matriz de grados de entrada o la matriz de grados de salida dependiendo de la aplicación que se esté considerando.

\begin{exampleth}
\end{exampleth}

\begin{theorem}
El Teorema de Kirchhoff. Sea un grafo conexo $G$ con $n$ vértices numerados y sean $\lambda_1$, $\lambda_2$, \dots, $\lambda_{n-1}$ los valores propios no nulos de su matriz laplaciana. Se tiene entonces que el número de árboles de expansión del grafo $G$ es

\begin{equation}
t(G) = \frac{1}{n} \cdot \lambda_1 \cdots \lambda_{n-1}
\end{equation}

Como podemos ver, se trata de una generalización de la fórmula de Cayley que además de muestra que el número de árboles de expansión de cualquier grafo se puede calcular en tiempo polinómico a partir del determinante de una submatriz de la matriz laplaciana. Específicamente, el número de árboles de expansión de un grafo conexo coincide con cualquier cofactor de su matriz laplaciana.
\end{theorem}

\ctparttext{
  \color{black}
  \begin{center}
    Análisis estadístico de los datos.
  \end{center}
}
\part{Análisis Descriptivo}

\chapter{Los registros existentes}
\addcontentsline{toc}{chapter}{Los registros existentes}

En el laboratorio remoto se ha registrado la actividad de $7$ años consecutivos (desde el curso académico $1516$ al $2122$). Un ejemplo de las interacciones almacenadas puede verse en la Tabla \ref{tab:example}.

\begin{table}[H]
\centering
\caption{Muestra de los datos que se recopilan en el servidor.}
\label{tab:example}
\begin{tabular}{cccccc}
\hline
\textbf{Year} & \textbf{Group} & \textbf{SessionID} & \textbf{Date}       & \textbf{Problem} & \textbf{Step} \\ \hline
1819          & Keid           & 493252533735       & 28/10/2018 20:23:35 & P1               & 1             \\ 
1819          & Keid           & 493252533735       & 28/10/2018 20:23:40 & P1               & 3             \\ 
1819          & Keid           & 389034076811       & 7/11/2018 19:01:49  & P2               & 1             \\
1819          & Cerastes       & 487544594557       & 27/10/2018 13:05:11 & P1               & 1             \\
1819          & Cerastes       & 487544594557       & 27/10/2018 13:10:57 & P1               & 3             \\
1819          & Jabbah         & 550676318711       & 20/12/2018 22:22:42 & P8               & 1             \\
1819          & Cerastes       & 336303012053       & 17/12/2018 13:28:50 & P9               & 1             \\ 
1819          & Keid           & 563159878397       & 25/10/2018 12:41:43 & P8               & 1             \\ \hline
\end{tabular}
\end{table}

\section{Número de grupos cada año}

El número de grupos puede variar en cada curso en función del número de alumnos matriculados en la asignatura ese año. Así pues, se muestran a continuación en las Tablas \ref{tab:groups1} y \ref{tab:groups2} los grupos por curso académico.

\section{El periodo de tiempo analizado cada año}

En la Tabla \ref{tab:days} se muestra el número de días que dura la práctica cada año. Se puede apreciar que la duración de la práctica que estamos considerando puede variar en función del curso académico.

\begin{table}[H]
\centering
\caption{Número de días que dura la práctica cada año.}
\label{tab:days}
\begin{tabular}{cc}
\hline
\textbf{Year}  & \textbf{Length (days)}  \\ \hline
Y2015 & 33 \\
Y2016 & 24 \\
Y2017 & 30 \\
Y2018 & 18 \\
Y2019 & 28 \\
Y2020 & 17 \\
Y2021 & 39 \\ \hline
\end{tabular}
\end{table}

\section{El conjunto de problemas analizados cada año}

Todos los años hay 9 problemas de dificultad similar que deben ser resueltos por todos los grupos.

Para aproximarnos al concepto subjetivo de ``dificultad del problema'' vamos a analizar el número de sesiones fallidas que necesita cada alumno para resolverlos por primera vez con respecto al número total de sesiones de ese problema (tasa de fallo) y la duración de este periodo en horas.

\subsection{Dificultad del problema: la tasa de fallo}

La apertura de un problema se corresponde con una sesión de trabajo, la cual puede terminar como fallo (fail) si no se consigue resolver el problema, o éxito (solved) en caso de que se haya resuelto el problema. Así pues, se definirá la tasa de fallo como el cociente entre el número total de sesiones fallidas entre el número total de sesiones de un mismo problema. El boxplot de las tasas de fallo por problema puede verse en la Figura \ref{fig:boxplotfailratio}.

\begin{figure}[H]
    \centering
    \includegraphics[width=0.85\textwidth]{Rplot09.png}
    \caption{Boxplot de la tasa de fallo (Fail ratio) por problema.}
    \label{fig:boxplotfailratio}
\end{figure}

Tras realizar el test ANOVA de un factor (resultados en la Tabla \ref{tab:ANOVAfailratio}), cuya hipótesis nula establece que la tasa de fallo media de los nueve problemas considerados es la misma, se detecta que las distribuciones de probabilidad de la tasa de fallo son estadísticamente iguales en los distintos problemas ($p = 0.1733 > 0.05$)\footnote{Nótese que hemos establecido un nivel de significancia de $\alpha = 0.05$.}.

% latex table generated in R 4.3.0 by xtable 1.8-4 package
% Sat May 27 20:17:56 2023
\begin{table}[H]
\centering
\caption{Resultados del test ANOVA de un solo factor (tasa de fallo).}
\label{tab:ANOVAfailratio}
\begin{tabular}{lrrrrr}
  \hline
 & Df & Sum Sq & Mean Sq & F value & Pr($>$F) \\ 
  \hline
ndsp[[nVariable]] & 8 & 0.51 & 0.06 & 1.52 & 0.1733 \\ 
  Residuals         & 54 & 2.26 & 0.04 &  &  \\ 
   \hline
\end{tabular}
\end{table}

Además, se ha realizado un test de Tukey por pares de problemas (Tabla \ref{tab:Tukeyfailratio}). En él se observa que casi todos los pares pueden considerarse estadísticamente iguales ($\text{p adj} > 0.2$ en todos ellos) salvo quizá, el par P9-P5 ($\text{p adj} = 0.2$). La Figura \ref{fig:confidenceratiofail} muestra los intervalos de confianza de todas las diferencias entre las distintas parejas de años.

\begin{figure}[H]
    \centering
    \includegraphics[width=0.80\textwidth]{Rplot10.png}
    \caption{Intervalos de confianza de la tasas de fallo de los problemas.}
    \label{fig:confidenceratiofail}
\end{figure}

%Estos datos hay que interpretarlos de manera incremental, pues para resolver un problema Pi requiere las habilidades de Pj(0<=j<i) más habilidades nuevas propias de Pi Periódicamente se incrementa notablemente el nivel de dificultad. P1 es el que más cuesta porque siempre cuesta trabajo empezar, P2-4 son muy parecidos y P5 es un poco más difícil, P6-P8 suben un escalón de dificultad y P9 también.

\subsection{Dificultad del problema: tiempo necesario en resolverlo}

Es el número de horas que transcurren desde que el problema se abre por primera vez hasta que es resuelto por primera vez.

\textbf{Falta imagen.}

De nuevo los tests detectan comportamientos diferentes (ANOVA p=9.27e-5, KW p=8.8e-8).

\textbf{Falta tabla.}

Por pares.

\textbf{Falta tabla.}

Intervalos de confianza.

\textbf{Falta imagen.}

Por lo tanto, se puede ver, dadas las evidencias aportadas que la resolución de cada problema exige respuestas claramente diferentes por parte del alumnado.

\section{Actividad registrada}\label{sec:activityrecorded}

El número de registros y de sesiones de trabajo de cada uno de los años analizados se muestran en la Tabla \ref{tab:records}. Como podemos ver, aunque el curso académico $2021$ registra más actividad que los demás, no es el que presenta un mayor número de sesiones.

\begin{table}[H]
\centering
\caption{Número de registros y sesiones almacenados en el servidor por años.}
\label{tab:records}
\begin{tabular}{ccc}
\hline
\textbf{Year}  & \textbf{Activity Records} & \textbf{Sessions}  \\ \hline
Y2015 & 12088            &  4489  \\
Y2016 & 12525            &  4538  \\
Y2017 & 9088             &  3661  \\
Y2018 & 5705             &  2811  \\
Y2019 & 14475            &  5156  \\
Y2020 & 21188            &  3900  \\
Y2021 & 11961            &  6113  \\ \hline
\end{tabular}
\end{table}

Al ser un servicio 24 horas los 7 días de la semana, los alumnos interactúan con el laboratorio remoto en cualquier día de la semana tal y como puede verse en la Figura \ref{fig:days} y a cualquier hora del día (Figura \ref{fig:hours}).

\begin{figure}[H]
\centering
\subfloat[Histograma de los días de la semana.]{\label{fig:days}\includegraphics[width=0.47\textwidth]{histogramdayofweek.png}}\qquad
\subfloat[Histograma de las horas del día.]{\label{fig:hours}\includegraphics[width=0.47\textwidth]{histogramhour.png}}
\caption{Actividad registrada en el servidor remoto.}
\label{fig:activity}
\end{figure}

El número y tipo de las sesiones de trabajo de cada uno de los grupos puede contemplarse en la Tabla \ref{tab:type}.

\subsection{Análisis de la normalidad de la distribución del número de sesiones}

En las Figuras \ref{fig:boxplotresiduals} y \ref{fig:histogramresiduals} podemos ver el boxplot de los residuos y el histograma de los mismos.

\begin{figure}[H]
\centering
\subfloat[Boxplot de los residuos del número de sesiones.]{\label{fig:boxplotresiduals}\includegraphics[width=0.47\textwidth]{Rplot02.png}}\qquad
\subfloat[Histograma de los residuos del número de sesiones.]{\label{fig:histogramresiduals}\includegraphics[width=0.47\textwidth]{Rplot03.png}}
\caption{Distribución de los residuos del número de sesiones.}
\label{fig:activity}
\end{figure}

A continuación, en las Figuras \ref{fig:densitysessions} y \ref{fig:q-qsessions}, podemos observar que la distribución del número de sesiones no es perfectamente normal pero es casi-normal si eliminaremos algunos outsiders.

\begin{figure}[H]
    \centering
    \includegraphics[width=0.75\textwidth]{Rplot04.png}
    \caption{Función de densidad de probabilidad del número de sesiones.}
    \label{fig:densitysessions}
\end{figure}


\begin{figure}[H]
    \centering
    \includegraphics[width=0.75\textwidth]{Rplot05.png}
    \caption{Gráfico Q-Q del número de sesiones.}
    \label{fig:q-qsessions}
\end{figure}

\subsection{Sesiones por cada problema}

En la Figura \ref{fig:boxplotsessionsproblem} podemos ver el boxplot del número de sesiones por problema. Como podemos ver, el problema P1 es mucho más frecuentado que el resto. No obstante, esto se debe la principal diferencia entre el número de sesiones abiertas del problema P1 y restantes se debe a que los alumnos utilizan el primer problema como base de todos los experimentos y para testear las comunicaciones con el servidor. Así pues, el problema P1 es frecuentemente utilizado, no ya sólo al comienzo, sino durante toda la práctica.

\begin{figure}[H]
    \centering
    \includegraphics[width=0.85\textwidth]{Rplot06.png}
    \caption{Boxplot del número de sesiones por problema.}
    \label{fig:boxplotsessionsproblem}
\end{figure}

\subsection{Sesiones cada año}

Como podemos ver en la Figura \ref{fig:boxplotsessionsyear}, las sesiones de trabajo abiertas en el servidor año tras año, parecen seguir la misma distribución de probabilidad.

\begin{figure}[H]
    \centering
    \includegraphics[width=0.80\textwidth]{Rplot07.png}
    \caption{Boxplot del número de sesiones por año.}
    \label{fig:boxplotsessionsyear}
\end{figure}

Un resumen de los resultados obtenidos al realizar el test ANOVA se muestra en la Tabla \ref{tab:ANOVAnumsessions}. La hipótesis nula establece que el número de sesiones medio de los siete cursos académicos estudiados es el mismo. Así pues, estableciendo un nivel de significancia de $0.05$, como tenemos que $p = 0.0630 > 0.05$, las diferencias entre las medias no son estadísticamente significativas.

% latex table generated in R 4.3.0 by xtable 1.8-4 package
% Sat May 27 18:26:28 2023
\begin{table}[H]
\centering
\caption{Resultados del test ANOVA de un solo factor (número de sesiones).}
\label{tab:ANOVAnumsessions}
\begin{tabular}{lrrrrr}
  \hline
 & Df & Sum Sq & Mean Sq & F value & Pr($>$F) \\ 
  \hline
ndsp[[nVariable]] & 6 & 666498.27 & 111083.05 & 2.11 & 0.0630 \\ 
  Residuals         & 70 & 3688035.44 & 52686.22 &  &  \\ 
   \hline
\end{tabular}
\end{table}

Además, se ha realizado un test de Tukey por pares de años (Tabla \ref{tab:Tukeynumsessions}). En él se observa que todos los pares pueden considerarse estadísticamente iguales ($\text{p adj} > 0.2$ en todos ellos). La Figura \ref{fig:confidencenumsessions} muestra los intervalos de confianza de todas las diferencias entre las distintas parejas de años.

% latex table generated in R 4.3.0 by xtable 1.8-4 package
% Sat May 27 18:26:44 2023
\begin{table}[ht]
\centering
\caption{Test HSD de Tukey (Honestly-significance-difference) del número de sesiones por año.}
\label{tab:Tukeynumsessions}
\begin{tabular}{rrrrr}
  \hline
 & diff & lwr & upr & p adj \\ 
  \hline
Y2016-Y2015 & 5.44 & -323.03 & 333.92 & 1.00 \\ 
  Y2017-Y2015 & 24.22 & -326.93 & 375.38 & 1.00 \\ 
  Y2018-Y2015 & -243.23 & -556.42 & 69.96 & 0.23 \\ 
  Y2019-Y2015 & -69.11 & -376.37 & 238.15 & 0.99 \\ 
  Y2020-Y2015 & -198.78 & -500.93 & 103.38 & 0.43 \\ 
  Y2021-Y2015 & -116.72 & -407.05 & 173.62 & 0.88 \\ 
  Y2017-Y2016 & 18.78 & -332.38 & 369.93 & 1.00 \\ 
  Y2018-Y2016 & -248.68 & -561.87 & 64.51 & 0.21 \\ 
  Y2019-Y2016 & -74.56 & -381.82 & 232.71 & 0.99 \\ 
  Y2020-Y2016 & -204.22 & -506.38 & 97.93 & 0.39 \\ 
  Y2021-Y2016 & -122.16 & -412.49 & 168.18 & 0.86 \\ 
  Y2018-Y2017 & -267.45 & -604.35 & 69.45 & 0.21 \\ 
  Y2019-Y2017 & -93.33 & -424.73 & 238.06 & 0.98 \\ 
  Y2020-Y2017 & -223.00 & -549.67 & 103.67 & 0.38 \\ 
  Y2021-Y2017 & -140.94 & -456.70 & 174.83 & 0.82 \\ 
  Y2019-Y2018 & 174.12 & -116.74 & 464.98 & 0.54 \\ 
  Y2020-Y2018 & 44.45 & -241.01 & 329.92 & 1.00 \\ 
  Y2021-Y2018 & 126.52 & -146.40 & 399.44 & 0.80 \\ 
  Y2020-Y2019 & -129.67 & -408.61 & 149.28 & 0.79 \\ 
  Y2021-Y2019 & -47.60 & -313.70 & 218.49 & 1.00 \\ 
  Y2021-Y2020 & 82.06 & -178.12 & 342.24 & 0.96 \\ 
   \hline
\end{tabular}
\end{table}

\begin{figure}[H]
    \centering
    \includegraphics[width=0.80\textwidth]{Rplot08.png}
    \caption{Intervalos de confianza del número de sesiones por año.}
    \label{fig:confidencenumsessions}
\end{figure}
\chapter{Hipótesis de estudio}
\addcontentsline{toc}{chapter}{Hipótesis de estudio}

A pesar de que el estudio descriptivo anterior muestra unos datos muy variados, casi todos ellos son homogéneos año tras año. No obstante, el objetivo de este estudio es sentar las bases para conseguir una experiencia de aprendizaje óptima para todos los grupos de alumnos. Así pues, se va a poner énfasis en detectar a aquellos grupos que estén en riesgo de obtener un peor rendimiento o peores calificaciones. La detección temprana de éstos podría permitir al profesor actuar a tiempo para mejorar su proceso de aprendizaje. Para ello, se van a proponer una serie de métricas de calidad que se definirán sobre los registros de actividad de los alumnos con el objetivo de encontrar aquella que, con mayor certeza, identifique a los alumnos que peor están progresando.

\section{Métricas de calidad y correlaciones entre ellas}

Se definirán dos grandes grupos de métricas. El primer grupo consistirá en una colección de métricas de los grupos que solamente podrán calcularse tras la finalización de la práctica. Por el contrario, las métricas del segundo grupo podrán calcularse durante la realización de la práctica y, por tanto, serán más interesantes porque podrán facilitar la detección precoz de los grupos en riesgo.

\subsection{Medidas a posteriori del resultado de la práctica}

\begin{itemize}
\item \textbf{Achiever.} Se refiere a la nota (Grade) conseguida por el alumno. Obviamente, cuanto mayor sea ésta, mejor.
\item \textbf{Performer.} Número de problemas resueltos (Goals). Trivialmente, cuantos más objetivos haya resuelto un grupo, mejor.
\item \textbf{Terminator.} Punto de finalización de toda la práctica (Makespan). Cuanto antes, mejor (para disponer de más tiempo para repasar y corregir errores). Sin embargo, no es una métrica muy relevante.
\item \textbf{Timeburner.} Es el tiempo consumido por el alumno durante las prácticas (Duration). Este es un valor trampa, pues puede significar algo positivo (el alumno ha tardado poco en resolver la práctica porque la domina), o negativo (porque no ha podido dedicarle más tiempo).
\item \textbf{Perseverant.} Número de sesiones realizadas. Ya se ha hecho un estudio de esta medida en la sección \ref{sec:activityrecorded}.
\end{itemize}

\subsection{Medidas continuas durante la práctica}

\begin{itemize}
\item \textbf{SingleThreaded.} Cuantifica como de balanceado están los grafos que describen la actividad de un grupo en la plataforma. Además, se valora negativamente que los grupos vayan saltando de problema en problema (DAG).
\item \textbf{EarlyBird.} (FirstOpen) Como los comienzos son siempre costosos, se define  la medida EarlyBird como el promedio de tiempo de la primera apertura de cada problema.
\item \textbf{MessedUp.} Número de fails consecutivos (FailRatio) hasta resolver un problema dividido entre el numero de sesiones de ese problema. La tasa de fallo así como el tiempo dedicado a un mismo problema dependerá de la dificultad del mismo.
\item \textbf{Procastinator.} Esta medida cuantifica las posposición de tareas de los grupos. Es decir, pretende identificar a aquellos alumnos que, cuando intentan resolver un problema y no lo consiguen, saltan, curiosamente, a otros problemas más complejos (los cuales, obviamente, tampoco pueden resolver) perdiendo así un tiempo precioso.
\item \textbf{Follower.} Medida que refleja si los grupos de prácticas siguen el orden esperado de las mismas.
\end{itemize}
\chapter{Rendimiento observado de los alumnos}
\addcontentsline{toc}{chapter}{Rendimiento observado de los alumnos}

\section{Calificaciones obtenidas (Grade)}

La primera medida de rendimiento que tendremos en cuenta son las calificaciones de los grupos de prácticas. Hay que tener en cuenta que estas calificaciones no son la evaluación final de cada grupo, sino la nota de la práctica cuya evolución se está analizando. Esta es la parte más subjetiva de la evaluación del rendimiento de cada alumno pues implica la participación del profesor y la toma en consideración de otros factores, además de los registrados en el servidor como puede ser la calidad de la memoria de la práctica. Las calificaciones muestran una distribución normal a lo largo de estos siete años de registros, ligeramente inclinada a la derecha porque las notas medias de esta asignatura suelen ser altas tal y como puede verse en la Figura \ref{fig:densityplotachiever}.

\begin{figure}[H]
    \centering
    \includegraphics[width=0.70\textwidth]{rendimiento/densitygrade.png}
    \caption{Función de densidad de probabilidad de las calificaciones obtenidas por los distintos grupos de prácticas.}
    \label{fig:densityplotachiever}
\end{figure}

Además, la media está bastante balanceada (Figuras \ref{fig:boxplotresidualsachiever} y \ref{fig:histogramresidualsachiever}) y no se detecta la presencia de outliers (Figura \ref{fig:outliersgrade}).

\begin{figure}[H]
\centering
\subfloat[Boxplot de los residuos de las calificaciones.]{\label{fig:boxplotresidualsachiever}\includegraphics[width=0.47\textwidth]{rendimiento/residualsgrade.png}}\qquad
\subfloat[Histograma de los residuos de las calificaciones.]{\label{fig:histogramresidualsachiever}\includegraphics[width=0.47\textwidth]{rendimiento/histogramgrade.png}}
\caption{Distribución de los residuos de las calificaciones.}
\label{fig:achiever}
\end{figure}

\begin{figure}[H]
    \centering
    \includegraphics[width=0.60\textwidth]{rendimiento/outliersgrade.png}
    \caption{Distribución de las calificaciones obtenidas por los distintos grupos de alumnos inicial.}
    \label{fig:outliersgrade}
\end{figure}

Más aún, la regresión es muy aceptable (Figura \ref{fig:q-qsessionsachiever}).

\begin{figure}[H]
    \centering
    \includegraphics[width=0.75\textwidth]{rendimiento/qqplotgrade.png}
    \caption{Gráfico Q-Q de las calificaciones.}
    \label{fig:q-qsessionsachiever}
\end{figure}

A continuación se realizará un estudio por años de las calificaciones obtenidas. El boxplot de las calificaciones por años puede verse en la Figura \ref{fig:boxplotachieveryear}. Como puede verse, los datos recogidos muestran una variación muy perceptible en las notas a lo largo de los años. De hecho, los test estadísticos ratifican que hay diferentes significativas entre ellas. Tras realizar el test ANOVA de un factor (resultados en la Tabla \ref{tab:ANOVAachiever}) obtenemos $p = 0.00143 < 0.05$. Además, tras realizar el test de Kruskal-Wallis se obtiene $p-value = 0.01534$. Un análisis posterior por pares de Tukey muestra las diferencias entre los años (resultados en la Tabla \ref{tab:Tukeyachiever}). Podemos ver que el año 2017 es el principal elemento de disrupción, pero por poco margen. Lo mismo indica el análisis de los intervalos de confianza de la Figura \ref{fig:confidenceachiever}.

\begin{figure}[H]
    \centering
    \includegraphics[width=0.6\textwidth]{rendimiento/boxplotgrade.png}
    \caption{Boxplot de las calificaciones por año.}
    \label{fig:boxplotachieveryear}
\end{figure}

% latex table generated in R 4.3.0 by xtable 1.8-4 package
% Sun Jun  4 18:14:08 2023
\begin{table}[H]
\centering
\caption{Resultados del test ANOVA de un solo factor (calificaciones).}
\label{tab:ANOVAachiever}
% latex table generated in R 4.3.0 by xtable 1.8-4 package
% Mon Jun  5 21:06:19 2023
\centering
\begin{tabular}{lrrrrr}
  \hline
 & Df & Sum Sq & Mean Sq & F value & Pr($>$F) \\ 
  \hline
rdataset[[Variable]] & 6 & 13.05 & 2.18 & 4.11 & 0.0014 \\ 
  Residuals            & 67 & 35.47 & 0.53 &  &  \\ 
   \hline
\end{tabular}
\end{table}

% latex table generated in R 4.3.0 by xtable 1.8-4 package
% Sun Jun  4 18:14:38 2023
\begin{table}[H]
\centering
\caption{Test HSD de Tukey (Honestly-significance-difference) de las calificaciones por años.}
\label{tab:Tukeyachiever}
\begin{tabular}{rrrrr}
  \hline
 & diff & lwr & upr & p adj \\ 
  \hline
Y2016-Y2015 & 0.18 & -0.87 & 1.22 & 1.00 \\ 
  Y2017-Y2015 & -1.35 & -2.52 & -0.19 & 0.01 \\ 
  Y2018-Y2015 & -0.45 & -1.47 & 0.57 & 0.83 \\ 
  Y2019-Y2015 & -0.24 & -1.23 & 0.76 & 0.99 \\ 
  Y2020-Y2015 & -0.10 & -1.06 & 0.85 & 1.00 \\ 
  Y2021-Y2015 & 0.22 & -0.70 & 1.14 & 0.99 \\ 
  Y2017-Y2016 & -1.53 & -2.70 & -0.36 & 0.00 \\ 
  Y2018-Y2016 & -0.63 & -1.64 & 0.39 & 0.51 \\ 
  Y2019-Y2016 & -0.41 & -1.41 & 0.58 & 0.87 \\ 
  Y2020-Y2016 & -0.28 & -1.24 & 0.68 & 0.97 \\ 
  Y2021-Y2016 & 0.05 & -0.88 & 0.97 & 1.00 \\ 
  Y2018-Y2017 & 0.90 & -0.24 & 2.05 & 0.21 \\ 
  Y2019-Y2017 & 1.12 & -0.00 & 2.24 & 0.05 \\ 
  Y2020-Y2017 & 1.25 & 0.16 & 2.34 & 0.01 \\ 
  Y2021-Y2017 & 1.57 & 0.52 & 2.63 & 0.00 \\ 
  Y2019-Y2018 & 0.21 & -0.75 & 1.18 & 0.99 \\ 
  Y2020-Y2018 & 0.34 & -0.59 & 1.28 & 0.92 \\ 
  Y2021-Y2018 & 0.67 & -0.22 & 1.56 & 0.27 \\ 
  Y2020-Y2019 & 0.13 & -0.78 & 1.04 & 1.00 \\ 
  Y2021-Y2019 & 0.46 & -0.41 & 1.32 & 0.68 \\ 
  Y2021-Y2020 & 0.33 & -0.50 & 1.15 & 0.89 \\ 
   \hline
\end{tabular}
\end{table}

\begin{figure}[H]
    \centering
    \includegraphics[width=0.80\textwidth]{rendimiento/confidencegrade.png}
    \caption{Intervalos de confianza de las calificaciones por años.}
    \label{fig:confidenceachiever}
\end{figure}

\section{Número de sesiones realiazadas (Perseverant)}

El número de sesiones realizadas por los diferentes grupos de prácticas se ha analizado con anterioridad. En la subsección \ref{sec:NormalityNumSessions} vimos que seguía una distribución casi-normal. Además, vimos que el número de sesiones sigue la misma distribución de probabilidad en cada uno de los cursos académicos considerados (subsección \ref{sec:ANOVANumSessions}).

\section{Número de intentos para resolver cada problema (SessionsBefore)}

Número de veces que se abre un mismo problema sin resolver hasta que se resuelve por primera vez. Este valor está directamente relacionado con la tasa de fallo analizada anteriormente. Podemos ver que esta medida de rendimiento sigue una distribución casi normal, ligeramente inclinada a la izquierda tal y como puede verse en la Figura \ref{fig:densityplotsessionsbefore}.

\begin{figure}[H]
    \centering
    \includegraphics[width=0.70\textwidth]{densitysessionsbefore.png}
    \caption{Función de densidad de probabilidad del número de intentos necesarios para resolver un problema por los distintos grupos de prácticas.}
    \label{fig:densityplotsessionsbefore}
\end{figure}

Sin embargo, podemos ver algunos grupos que emplean muy pocos intentos para resolver un problema o que, por el contrario, emplean más intentos que los demás (Figuras \ref{fig:boxplotresidualssessionsbefore} y \ref{fig:histogramresidualssessionsbefore}).

\begin{figure}[H]
\centering
\subfloat[Boxplot de los residuos del número de intentos.]{\label{fig:boxplotresidualssessionsbefore}\includegraphics[width=0.47\textwidth]{NormalitySessionsBefore3.png}}\qquad
\subfloat[Histograma de los residuos del número de intentos.]{\label{fig:histogramresidualssessionsbefore}\includegraphics[width=0.47\textwidth]{NormalitySessionsBefore4.png}}
\caption{Distribución de los residuos del número de intentos.}
\label{fig:sessionsbefore}
\end{figure}

No obstante, podemos considerar que la regresión es aceptable (Figura \ref{fig:q-qsessionsbefore}).

\begin{figure}[H]
    \centering
    \includegraphics[width=0.75\textwidth]{NormalitySessionsBefore6.png}
    \caption{Gráfico Q-Q del número de intentos.}
    \label{fig:q-qsessionsbefore}
\end{figure}

A continuación, se realizará un estudio por años del número de intentos antes de resolver un problema realizados. Como podemos observar en la Figura \ref{fig:boxplotsessionsbefore}, hay una variación muy perceptible en el número de intentos realizados a lo largo de los años. La realización del test ANOVA de un factor (resultados en la Tabla \ref{tab:ANOVAsessionsbefore}) ratifica que hay diferencias significativas ($p = 0.0001 < 0.05$). Posteriormente, se ha realizado el test de Tukey por pares y se ha visto que hay diferencias entre las distribuciones de los años $2015$ y $2017$ son las que introducen diferencias (Tabla \ref{tab:Tukeysessionsbefore}). Esto mismo puede observarse en la Figura \ref{fig:confidencesessionsbefore}.

\begin{figure}[H]
    \centering
    \includegraphics[width=0.6\textwidth]{boxplotsessionsbefore.png}
    \caption{Boxplot del número de intentos para resolver un problema por año.}
    \label{fig:boxplotsessionsbefore}
\end{figure}

% latex table generated in R 4.3.0 by xtable 1.8-4 package
% Fri Jun  9 10:44:07 2023
\begin{table}[H]
\centering
\caption{Resultados del test ANOVA de un solo factor (calificaciones).}
\label{tab:ANOVAsessionsbefore}
\begin{tabular}{lrrrrr}
  \hline
 & Df & Sum Sq & Mean Sq & F value & Pr($>$F) \\ 
  \hline
rdataset[[Variable]] & 6 & 516318.38 & 86053.06 & 5.83 & 0.0001 \\ 
  Residuals            & 69 & 1018845.04 & 14765.87 &  &  \\ 
   \hline
\end{tabular}
\end{table}

% latex table generated in R 4.3.0 by xtable 1.8-4 package
% Fri Jun  9 10:44:35 2023
\begin{table}[H]
\centering
\caption{Test HSD de Tukey (Honestly-significance-difference) del número de intentos realizados para resolver un problema por años.}
\label{tab:Tukeysessionsbefore}
\begin{tabular}{rrrrr}
  \hline
 & diff & lwr & upr & p adj \\ 
  \hline
Y2016-Y2015 & -95.44 & -269.41 & 78.52 & 0.64 \\ 
  Y2017-Y2015 & 52.30 & -133.68 & 238.28 & 0.98 \\ 
  \textbf{Y2018-Y2015} & -164.56 & -330.43 & 1.32 & 0.05 \\ 
  Y2019-Y2015 & -111.37 & -277.25 & 54.50 & 0.40 \\ 
  \textbf{Y2020-Y2015} & -213.56 & -373.58 & -53.53 & 0.00 \\ 
  \textbf{Y2021-Y2015} & -168.24 & -322.01 & -14.48 & 0.02 \\ 
  Y2017-Y2016 & 147.75 & -38.23 & 333.73 & 0.21 \\ 
  Y2018-Y2016 & -69.11 & -234.98 & 96.76 & 0.87 \\ 
  Y2019-Y2016 & -15.93 & -181.80 & 149.94 & 1.00 \\ 
  Y2020-Y2016 & -118.11 & -278.14 & 41.92 & 0.29 \\ 
  Y2021-Y2016 & -72.80 & -226.57 & 80.97 & 0.78 \\ 
  \textbf{Y2018-Y2017} & -216.86 & -395.29 & -38.43 & 0.01 \\ 
  \textbf{Y2019-Y2017} & -163.68 & -342.10 & 14.75 & 0.09 \\ 
  \textbf{Y2020-Y2017} & -265.86 & -438.87 & -92.85 & 0.00 \\ 
  \textbf{Y2021-Y2017} & -220.54 & -387.78 & -53.31 & 0.00 \\ 
  Y2019-Y2018 & 53.18 & -104.18 & 210.54 & 0.95 \\ 
  Y2020-Y2018 & -49.00 & -200.19 & 102.19 & 0.96 \\ 
  Y2021-Y2018 & -3.69 & -148.23 & 140.86 & 1.00 \\ 
  Y2020-Y2019 & -102.18 & -253.37 & 49.00 & 0.39 \\ 
  Y2021-Y2019 & -56.87 & -201.41 & 87.68 & 0.89 \\ 
  Y2021-Y2020 & 45.31 & -92.49 & 183.11 & 0.95 \\ 
   \hline
\end{tabular}
\end{table}

\begin{figure}[H]
    \centering
    \includegraphics[width=0.80\textwidth]{confidencesessionsbefore.png}
    \caption{Intervalos de confianza del número de intentos antes de resolver un problema por años.}
    \label{fig:confidencesessionsbefore}
\end{figure}

\section{Sesiones perdidas durante un problema}

Con frecuencia ocurre que los alumnos, cuando intentan resolver un problema y no lo consiguen, saltan, curiosamente, a sesiones en otros problemas más complejos, los cuales, obviamente, tampoco pueden resolver.

\textbf{Falta gráfica.}

Este hábito es más frecuente de lo que parece y mantiene mucha variación en cada problema, sobre todo es más fuerte al comienzo de las prácticas, pero es homogéneo año tras año (ANOVA p=0.746, KW p=0.9) y se puede decir que está presente en la mayoría de los grupos. Aunque la mediana sea 0, casi todos los grupos (80\%) exhiben este comportamiento en algún momento.

\textbf{Referenciar tabla ya existente}.

\textbf{Falta gráfica.}

\section{Abrir un problema por primera vez (Newcomer)}

Momento exacto en el que se consigue abrir cada problema por primera vez en el servidor, normalizado para poder compararlo (normalizado porque la duración de la práctica no es la misma todos los años). Como podemos ver en su función de densidad (Figura \ref{fig:densityplotnewcomer}), se aproxima a una distribución normal, un poco ladeada hacia la derecha. Esto puede deberse a que los distintos equipos tienen a abrir los problemas cuando se va aproximando la fecha de entrega de la práctica.

\begin{figure}[H]
    \centering
    \includegraphics[width=0.70\textwidth]{densitynewcomer.png}
    \caption{Función de densidad de probabilidad del momento en el que los distintos grupos de prácticas abren por primera vez un problema.}
    \label{fig:densityplotnewcomer}
\end{figure}

En las Figuras \ref{fig:boxplotresidualsnewcomer} y \ref{fig:histogramresidualsnewcomer} pueden verse el boxplot de los residuos y el histograma de los mismos respectivamente.

\begin{figure}[H]
\centering
\subfloat[Boxplot de los residuos del número de intentos.]{\label{fig:boxplotresidualsnewcomer}\includegraphics[width=0.47\textwidth]{NormalityNewcomer3.png}}\qquad
\subfloat[Histograma de los residuos del momento en el que se abren los problemas por primera vez.]{\label{fig:histogramresidualsnewcomer}\includegraphics[width=0.47\textwidth]{NormalityNewcomer4.png}}
\caption{Distribución de los residuos del momento en el que se abren los problemas por primera vez.}
\label{fig:newcomer}
\end{figure}

Si realizamos una segmentación por años, podemos ver que la media de esta medida de rendimiento puede variar según el año que estemos considerando (Figura \ref{fig:boxplotnewcomer}). El Test ANOVA de un sólo factor ha confirma lo observado (se ha obtenido $p = 0.0024 < 0.05$ como puede verse en la Tabla \ref{tab:ANOVAnewcomer}).

\begin{figure}[H]
    \centering
    \includegraphics[width=0.6\textwidth]{boxplotnewcomer.png}
    \caption{Boxplot del número del momento en el que se abre un problema por primera vez por años.}
    \label{fig:boxplotnewcomer}
\end{figure}

% latex table generated in R 4.3.0 by xtable 1.8-4 package
% Sun Jun 11 18:57:16 2023
\begin{table}[H]
\centering
\caption{Resultados del test ANOVA de un solo factor (momento en el que se abre un problema por primera vez).}
\label{tab:ANOVAnewcomer}
\begin{tabular}{lrrrrr}
  \hline
 & Df & Sum Sq & Mean Sq & F value & Pr($>$F) \\ 
  \hline
rdataset[[Variable]] & 6 & 0.71 & 0.12 & 3.82 & 0.0024 \\ 
  Residuals            & 69 & 2.12 & 0.03 &  &  \\ 
   \hline
\end{tabular}
\end{table}

Tras realizar el Test de Tukey por pares se ha visto que claramente hay años con distribuciones diferentes de esta variable (Tabla \ref{tab:Tukeynewcomer} y Figura \ref{fig:confidencenewcomer}).

% latex table generated in R 4.3.0 by xtable 1.8-4 package
% Sun Jun 11 18:57:48 2023
\begin{table}[H]
\centering
\caption{Test HSD de Tukey (Honestly-significance-difference) del momento exacto en el que se abre un problema por primera por años.}
\label{tab:Tukeynewcomer}
\begin{tabular}{rrrrr}
  \hline
 & diff & lwr & upr & p adj \\ 
  \hline
Y2016-Y2015 & 0.05 & -0.21 & 0.30 & 1.00 \\ 
  Y2017-Y2015 & 0.03 & -0.24 & 0.29 & 1.00 \\ 
  Y2018-Y2015 & -0.25 & -0.49 & -0.01 & 0.03 \\ 
  Y2019-Y2015 & -0.17 & -0.41 & 0.07 & 0.30 \\ 
  Y2020-Y2015 & -0.07 & -0.30 & 0.16 & 0.97 \\ 
  Y2021-Y2015 & -0.08 & -0.31 & 0.14 & 0.91 \\ 
  Y2017-Y2016 & -0.02 & -0.29 & 0.25 & 1.00 \\ 
  Y2018-Y2016 & -0.30 & -0.54 & -0.06 & 0.01 \\ 
  Y2019-Y2016 & -0.22 & -0.46 & 0.02 & 0.09 \\ 
  Y2020-Y2016 & -0.11 & -0.35 & 0.12 & 0.74 \\ 
  Y2021-Y2016 & -0.13 & -0.35 & 0.09 & 0.56 \\ 
  Y2018-Y2017 & -0.28 & -0.54 & -0.02 & 0.03 \\ 
  Y2019-Y2017 & -0.20 & -0.46 & 0.06 & 0.24 \\ 
  Y2020-Y2017 & -0.09 & -0.34 & 0.16 & 0.91 \\ 
  Y2021-Y2017 & -0.11 & -0.35 & 0.13 & 0.81 \\ 
  Y2019-Y2018 & 0.08 & -0.15 & 0.31 & 0.94 \\ 
  Y2020-Y2018 & 0.18 & -0.03 & 0.40 & 0.16 \\ 
  Y2021-Y2018 & 0.17 & -0.04 & 0.38 & 0.19 \\ 
  Y2020-Y2019 & 0.10 & -0.11 & 0.32 & 0.77 \\ 
  Y2021-Y2019 & 0.09 & -0.12 & 0.30 & 0.85 \\ 
  Y2021-Y2020 & -0.02 & -0.21 & 0.18 & 1.00 \\ 
   \hline
\end{tabular}
\end{table}

\begin{figure}[H]
    \centering
    \includegraphics[width=0.80\textwidth]{confidencenewcomer.png}
    \caption{Intervalos de confianza del momento en el que se abre un problema por años.}
    \label{fig:confidencenewcomer}
\end{figure}


\section{Resolver un problema por primera vez (EarlyBird)}

Momento exacto en el que se consigue resolver cada problema por primera vez, normalizado para poder compararlo (porque la duración de la práctica no es la misma todos los años). Como podemos ver en la Figura \ref{fig:densityplotearlybird}, sigue una distribución casi normal, ladeada hacia la derecha ya que los distintos grupos tienden a resolver los problemas por primera vez al final de la práctica.

\begin{figure}[H]
    \centering
    \includegraphics[width=0.70\textwidth]{densityearlybird.png}
    \caption{Función de densidad de probabilidad del momento en el que los distintos grupos de prácticas resuelven por primera vez un problema.}
    \label{fig:densityplotearlybird}
\end{figure}

Sin embargo, podemos observar la presencia de algunos outliers (Figuras \ref{fig:boxplotresidualsearlybird} y \ref{fig:histogramresidualsearlybird}).

\begin{figure}[H]
\centering
\subfloat[Boxplot de los residuos del número de intentos.]{\label{fig:boxplotresidualsearlybird}\includegraphics[width=0.47\textwidth]{NormalityEarlyBird3.png}}\qquad
\subfloat[Histograma de los residuos del momento en el que se resuelven los problemas por primera vez.]{\label{fig:histogramresidualsearlybird}\includegraphics[width=0.47\textwidth]{NormalityEarlyBird4.png}}
\caption{Distribución de los residuos del momento en el que se resuelven los problemas por primera vez.}
\label{fig:earlybird}
\end{figure}

Si se realiza una segmentación por años, intuimos que esta medida de rendimiento no sigue la misma distribución de probabilidad todos los cursos académicos estudiados (Figura \ref{fig:boxplotearlybird}). Esto se confirma tras la realización del Test ANOVA de un solo factor (Tabla \ref{tab:ANOVAearlybird}) en el que obtenemos $p = 0.0053 < 0.05$. Además, el Test de Tukey muestra las diferencias entre los años considerados (Tabla \ref{tab:Tukeyearlybird}).

\begin{figure}[H]
    \centering
    \includegraphics[width=0.6\textwidth]{boxplotearlybird.png}
    \caption{Boxplot del momento en el que se resuelve un problema por primera vez por años.}
    \label{fig:boxplotearlybird}
\end{figure}

% latex table generated in R 4.3.0 by xtable 1.8-4 package
% Sun Jun 11 19:03:22 2023
\begin{table}[H]
\centering
\caption{Resultados del test ANOVA de un solo factor (momento en el que se resuelve un problema por primera vez).}
\label{tab:ANOVAearlybird}
\begin{tabular}{lrrrrr}
  \hline
 & Df & Sum Sq & Mean Sq & F value & Pr($>$F) \\ 
  \hline
rdataset[[Variable]] & 6 & 0.50 & 0.08 & 3.40 & 0.0053 \\ 
  Residuals            & 69 & 1.70 & 0.02 &  &  \\ 
   \hline
\end{tabular}
\end{table}

% latex table generated in R 4.3.0 by xtable 1.8-4 package
% Sun Jun 11 19:03:51 2023
\begin{table}[H]
\centering
\caption{Test HSD de Tukey (Honestly-significance-difference) del momento en el que se resuelve un problema por primera vez, por años.}
\label{tab:Tukeyearlybird}
\begin{tabular}{rrrrr}
  \hline
 & diff & lwr & upr & p adj \\ 
  \hline
Y2016-Y2015 & 0.04 & -0.19 & 0.26 & 1.00 \\ 
  Y2017-Y2015 & 0.03 & -0.21 & 0.27 & 1.00 \\ 
  Y2018-Y2015 & -0.13 & -0.35 & 0.08 & 0.51 \\ 
  Y2019-Y2015 & -0.12 & -0.33 & 0.10 & 0.65 \\ 
  Y2020-Y2015 & -0.18 & -0.39 & 0.02 & 0.12 \\ 
  Y2021-Y2015 & -0.14 & -0.34 & 0.05 & 0.31 \\ 
  Y2017-Y2016 & -0.01 & -0.25 & 0.24 & 1.00 \\ 
  Y2018-Y2016 & -0.17 & -0.38 & 0.05 & 0.22 \\ 
  Y2019-Y2016 & -0.15 & -0.37 & 0.06 & 0.33 \\ 
  Y2020-Y2016 & -0.22 & -0.43 & -0.01 & 0.03 \\ 
  Y2021-Y2016 & -0.18 & -0.38 & 0.02 & 0.10 \\ 
  Y2018-Y2017 & -0.16 & -0.39 & 0.07 & 0.34 \\ 
  Y2019-Y2017 & -0.15 & -0.38 & 0.08 & 0.46 \\ 
  Y2020-Y2017 & -0.21 & -0.44 & 0.01 & 0.07 \\ 
  Y2021-Y2017 & -0.17 & -0.39 & 0.04 & 0.19 \\ 
  Y2019-Y2018 & 0.02 & -0.19 & 0.22 & 1.00 \\ 
  Y2020-Y2018 & -0.05 & -0.25 & 0.14 & 0.99 \\ 
  Y2021-Y2018 & -0.01 & -0.20 & 0.17 & 1.00 \\ 
  Y2020-Y2019 & -0.07 & -0.26 & 0.13 & 0.95 \\ 
  Y2021-Y2019 & -0.03 & -0.21 & 0.16 & 1.00 \\ 
  Y2021-Y2020 & 0.04 & -0.14 & 0.22 & 0.99 \\ 
   \hline
\end{tabular}
\end{table}

Por último, el análisis de los intervalos de confianza se muestra en la Figura \ref{fig:confidenceearlybird}.

\begin{figure}[H]
    \centering
    \includegraphics[width=0.80\textwidth]{confidenceearlybird.png}
    \caption{Intervalos de confianza del momento en el que se resuelve un problema por primera vez, por años.}
    \label{fig:confidenceearlybird}
\end{figure}

\section{Número total de problemas resueltos (p)}

Podemos observar que ha habido variaciones perceptibles durante los distintos años, con un caso especial en 2018 en el que hubo muchos grupos que no resolvieron todos los problemas (Figura \ref{fig:boxplotperformer}).

\begin{figure}[H]
    \centering
    \includegraphics[width=0.6\textwidth]{rendimiento/boxplotp.png}
    \caption{Boxplot del número de problemas resueltos por año.}
    \label{fig:boxplotperformer}
\end{figure}

Sin embargo, las diferencias entre las medias no son estadísticamente significativas (considerando un nivel de significancia de $0.05$) tal y como puede verse en la Tabla \ref{tab:ANOVAperformer}. Además, si consideramos el test estadístico de Kruskal-Wallis llegamos a la misma conclusión ($p-value = 0.365 > 0.05$). Así pues, se concluye que el número de problemas resueltos por años son uniformes (cualquier variación es debida al azar).

Además, realizando un test de Tukey por pares de años (Tabla \ref{tab:Tukeyperformer}) se observa que todos los pares pueden considerarse estadísticamente iguales. La Figura \ref{fig:confidenceperformer} muestra los intervalos de confianza de todas las diferencias entre las distintas parejas de años.

% latex table generated in R 4.3.0 by xtable 1.8-4 package
% Tue Jun  6 20:27:28 2023
\begin{table}[H]
\centering
\caption{Resultados del test ANOVA de un solo factor (número de problemas resueltos).}
\label{tab:ANOVAperformer}
\begin{tabular}{lrrrrr}
  \hline
 & Df & Sum Sq & Mean Sq & F value & Pr($>$F) \\ 
  \hline
rdataset[[Variable]] & 6 & 2.91 & 0.48 & 1.27 & 0.2835 \\ 
  Residuals            & 67 & 25.58 & 0.38 &  &  \\ 
   \hline
\end{tabular}
\end{table}

% latex table generated in R 4.3.0 by xtable 1.8-4 package
% Tue Jun  6 20:27:48 2023
\begin{table}[ht]
\centering
\caption{Test HSD de Tukey (Honestly-significance-difference) del número de problemas resueltos por año.}
\label{tab:Tukeyperformer}
\begin{tabular}{rrrrr}
  \hline
 & diff & lwr & upr & p adj \\ 
  \hline
Y2016-Y2015 & 0.22 & -0.66 & 1.11 & 0.99 \\ 
  Y2017-Y2015 & 0.22 & -0.77 & 1.21 & 0.99 \\ 
  Y2018-Y2015 & -0.04 & -0.91 & 0.82 & 1.00 \\ 
  Y2019-Y2015 & 0.28 & -0.56 & 1.13 & 0.95 \\ 
  Y2020-Y2015 & -0.29 & -1.11 & 0.52 & 0.93 \\ 
  Y2021-Y2015 & 0.18 & -0.60 & 0.96 & 0.99 \\ 
  Y2017-Y2016 & 0.00 & -0.99 & 0.99 & 1.00 \\ 
  Y2018-Y2016 & -0.27 & -1.13 & 0.60 & 0.96 \\ 
  Y2019-Y2016 & 0.06 & -0.78 & 0.90 & 1.00 \\ 
  Y2020-Y2016 & -0.51 & -1.33 & 0.30 & 0.48 \\ 
  Y2021-Y2016 & -0.04 & -0.82 & 0.74 & 1.00 \\ 
  Y2018-Y2017 & -0.27 & -1.24 & 0.70 & 0.98 \\ 
  Y2019-Y2017 & 0.06 & -0.89 & 1.01 & 1.00 \\ 
  Y2020-Y2017 & -0.51 & -1.44 & 0.41 & 0.63 \\ 
  Y2021-Y2017 & -0.04 & -0.94 & 0.86 & 1.00 \\ 
  Y2019-Y2018 & 0.33 & -0.49 & 1.15 & 0.89 \\ 
  Y2020-Y2018 & -0.25 & -1.04 & 0.54 & 0.96 \\ 
  Y2021-Y2018 & 0.22 & -0.53 & 0.98 & 0.97 \\ 
  Y2020-Y2019 & -0.57 & -1.34 & 0.20 & 0.28 \\ 
  Y2021-Y2019 & -0.10 & -0.84 & 0.63 & 1.00 \\ 
  Y2021-Y2020 & 0.47 & -0.23 & 1.17 & 0.40 \\ 
   \hline
\end{tabular}
\end{table}

\begin{figure}[H]
    \centering
    \includegraphics[width=0.80\textwidth]{rendimiento/confidencep.png}
    \caption{Intervalos de confianza del número de problemas resueltos por año.}
    \label{fig:confidenceperformer}
\end{figure}

Por último, podemos concluir que la variable número de problemas resueltos por año (\emph{p}) no es muy normal en tanto que varía muy poco y es discreta (Figuras \ref{fig:countp} y \ref{fig:densityp}).

\begin{figure}[H]
\centering
\subfloat[Número de grupos que han resuelto una determinada cantidad de problemas.]{\label{fig:countp}\includegraphics[width=0.47\textwidth]{rendimiento/countp.png}}\qquad
\subfloat[Función de densidad de probabilidad del número de problemas resueltos.]{\label{fig:densityp}\includegraphics[width=0.47\textwidth]{rendimiento/densityp.png}}
\caption{Distribución del número de problemas resueltos por cada grupo de alumnos.}
\label{fig:normalityp}
\end{figure}

\section{Siguiendo el plan del profesor (Follower)}

Se incorpora una medida de similaridad Follower en [0,1] que cuantifica cómo se parece el patrón encontrado con respecto al patrón esperado. \textbf{Falta footnote.}

\textbf{Falta tabla.}
\chapter{Características topológicas de los grafos de procesos}
\addcontentsline{toc}{chapter}{Características topológicas de los grafos de procesos}

Teniendo en mente que la finalidad última de este trabajo es encontrar indicadores del progreso de los alumnos y la detección precoz de aquellos grupos con problemas para que el profesor pueda proporcionarles ayuda y orientación, se han obtenido medidas de rendimiento sobre los procesos minados. Se tratarán de funciones \emph{off-the-shelf} genéricas sobre grafos, lo que permitiría aplicar los resulados aquí obtenidos en otras plataformas educativas.

Así pues, se han definido dos métricas distintas: el Laplaciano (\emph{Laplacian}), que usará el análisis espectral de grafos, y la heurística \emph{DAG}, que trata de determinar cómo de balanceados están los nodos de un grafo dirigido acíclico.

\subsection{El Laplaciano (Laplacian)}

En primer lugar, empezamos analizando el \emph{Learning Path} de los grupos para tener una idea de los problemas por los que ha ido navegando durante toda la práctica. En la Figura Figura \ref{fig:DBA1516P2GG} podemos ver el recorrido que hizo el grupo \texttt{DBA1920P2GG}. No obstante, para el cálculo de este coeficiente se usará un grafo de mayor complejidad, en el que se subdividen los estados en \texttt{Pi OK} o \texttt{Pi FAIL} dependiendo del milestone alcanzado (\texttt{OK} indica que se ha resuelto problema). En la Figura \ref{fig:DBA1516P2GG_states} se muestra este nuevo grupo para el grupo de prácticas que estamos considerando.

\begin{figure}[H]
\centering
\subfloat[Grafo que muestra la exploración de los problemas que ha realizado.]{\label{fig:DBA1516P2GG}\includegraphics[width=0.47\textwidth]{DBA1516P2GG.png}}\qquad
\subfloat[Grafo que muestra la exploración de los problemas, considerando si un problema ha sido resuelto (\texttt{OK}) o no (\texttt{FAIL}).]{\label{fig:DBA1516P2GG_states}\includegraphics[width=0.47\textwidth]{DBA1516P2GG_states.png}}
\caption{Leaning Path del grupo de prácticas \texttt{DBA1516P2GG}.}
\label{fig:laplacian}
\end{figure}

\ctparttext{
  \color{black}
  \begin{center}
    Estructuración en objetivos, división en sprints y seguimiento.
  \end{center}
}
\part{Planificación del proyecto}

\documentclass[10pt,a4paper]{article}
\usepackage[utf8]{inputenc}
\usepackage{amsmath}
\usepackage{amsfonts}
\usepackage{amssymb}
\usepackage{graphicx}
\usepackage[hidelinks]{hyperref} 
\usepackage{color}
\usepackage{xcolor}
\usepackage{caption}
\usepackage{subcaption}
\author{María Isabel Ruiz Martínez}
\title{Objetivos}

%Ruta absoluta en formato tipo Unix (Linux, OsX)
\graphicspath{ {/home/maribel/Escritorio/5º DGIIM/TFG/Analysis-of-processes/documentation/images} }

\begin{document}

\maketitle

Tras un primer análisis de los datos, se pretende tratar de relevar posibles estrategias escondidas en los mismos utilizando técnicas de minería de procesos, donde se considerará que una estrategia es el proceso seguido por los alumnos hasta que llegan a su objetivo.

Es decir, se prentende identificar patrones de comportamiento de los drones en los diferentes mundos virtuales con la finalidad de ayudar al profesorado de la asignatura a guiar al alumnado.

\end{document}
\chapter{Etapas del proyecto: división en sprints y seguimiento de los mismos}\label{chapter:sprints}
\addcontentsline{toc}{chapter}{Etapas del proyecto: división en sprints}

El proyecto se ha dividido en diez sprints de dos semanas cada uno. Además, durante el transcurso de cada sprint se ha realizado un seguimiento del trabajo realizado en el mismo, aportando \emph{burndown charts} de cada uno de ellos.

Un \emph{burdown chart} es una gráfica en la que se muestra el progreso de un proyecto durante cierto periodo de tiempo preestablecido (un sprint, una release o el proyecto completo, por ejemplo). Para la construcción de los mismos se requieren los siguientes elementos:
\begin{itemize}
\item El periodo de \emph{tiempo a analizar}, que se corresponderá con el eje X de la gráfica. El inicio del periodo temporal vendrá representado por $x = 0$ mientras que el final del periodo temporal será representado por $x = t$ donde $t$ es la duración del periodo. Si observamos el burndown chart global del proyecto (Figura global), estos dos valores se corresponderán, respectivamente, con las fechas de inicio y finalización del mismo.
\item La cantidad de trabajo a realizar, que se corresponderá con el eje Y de la gráfica y representa el trabajo planificado que se deberá realizar en el periodo de tiempo a analizar. Cuando se realiza la estimación de las tareas, independientemente de la unidad de estimación empleada, se obtiene una cantidad de trabajo a realizar (en horas, por ejemplo). Así pues, la cantidad de trabajo restante irá decreciendo conforme el tiempo vaya avanzando.
\item Una referencia ideal, que será la línea diagonal trazada desde la esquina superior izquierda hasta la esquina inferior derecha de la gráfica. Se trata de una representación de la relación ideal entra la disminución de la cantidad de trabajo y el tiempo dedicado durante el transcurso de la fase correspondiente. Esto es, cuanto más se aproxime la línea real a la ideal se estará trabajando de mejor manera en relación a la consecución de los objetivos marcados.
\end{itemize}

Adicionalmente, si se poseen conocimientos y experiencia trabajando con herramientas ágiles, se puede obtener información útil sobre el seguimiento del proyecto con el fin de impulsar buenas prácticas e impulsar el desarrollo del mismo.

\section{Análisis de cada sprint}

\begin{figure}[H]
\centering
\subfloat[Burndown chart del sprint 1 (periodo del 12/12/2022 al 01/01/2023).]{\label{fig:sprint1}\includegraphics[width=0.46\textwidth]{sprints/Burndown Sprint 1.png}}\qquad
\subfloat[Burndown chart del sprint 2 (periodo del 02/01/2023 al 22/01/2023).]{\label{fig:sprint2}\includegraphics[width=0.46\textwidth]{sprints/Burndown Sprint 2.png}}\qquad
\subfloat[Burndown chart del sprint 3 (periodo del 23/01/2023 al 12/02/2023).]{\label{fig:sprint3}
\includegraphics[width=0.46\textwidth]{sprints/Burndown Sprint 3.png}}\qquad
\subfloat[Burndown chart del sprint 4 (periodo del 13/02/2023 al 05/03/2023).]{\label{fig:sprint4}
\includegraphics[width=0.46\textwidth]{sprints/Burndown Sprint 4.png}}\qquad
\subfloat[Burndown chart del sprint 5 (periodo del 06/03/2023 al 26/03/2023).]{\label{fig:sprint5}
\includegraphics[width=0.46\textwidth]{sprints/Burndown Sprint 5.png}}\qquad
\subfloat[Burndown chart del sprint 6 (periodo del 27/03/2023 al 16/04/2023).]{\label{fig:sprint6}
\includegraphics[width=0.46\textwidth]{sprints/Burndown Sprint 6.png}}
\caption{Burdown charts de los seis primeros sprints del proyecto.}
\label{fig:sprints1-6}
\end{figure}

\subsection{Sprint 1 (Figura \ref{fig:sprint1})}

Este primer sprint representa una situación anómala que no se debería dar. La explicación es sencilla: las líneas rojas horizontales representan periodos de tiempo en los cuales no se ha estado trabajando en el proyecto. No obstante, esto es correcto puesto que aunque se definieron unos objetivos al principio, todavía no tenía claro el procedimiento a seguir para la consecución de los mismos. Aunque podría haberse omitido este periodo de seguimiento, se ha decidio incluirlo como muestra de una tendencia no deseable en relación al desarrollo de un proyecto.

\subsection{Sprint 2 (Figura \ref{fig:sprint2})}

En este segundo sprint tampoco se han alcanzado los objetivos de trabajo preestablecidos. No obstante, es correcto porque este periodo coincide con la realización de exámenes y se planificó previamente para el estudio de los mismos y no para continuar con la realización de este trabajo fin de grado. Como se ha comentado anteriormente, durante el desarrollo de un proyecto, esta situación no debería ocurrir bajo ninguna circunstancia.

\subsection{Sprint 3 (Figura \ref{fig:sprint3})}

En este caso, la Figura \ref{fig:sprint3} muestra que se realizó una gran cantidad de trabajo focalizada en días concretos distribuidos a lo largo de todo el sprint con algunos parones entre medias. Sin embargo, a pesar de todo, no se consiguió llegor a la cantidad estipulada de trabajo al final del sprint.

\subsection{Sprint 4 (Figura \ref{fig:sprint4})}

Este cuarto sprint consiste en un periodo inicial inactivo seguido de varios picos de productividad en días concretos del sprint. Finalmente, podemos observar que se realizó una gran cantidad de trabajo al final del sprint, quedándose muy cerca la cantidad de trabajo realizada de la referencia ideal.

\subsection{Sprint 5 (Figura \ref{fig:sprint5})}

En este sprint podemos ver que se realizó más trabajo del requerido durante el primer tercio del mismo. Sin embergo, a partir de entonces, a pesar de que se avanza en días concretos, el ritmo de trabajo es mucho más lento y se pasa a no realizar la cantidad de trabajo planificado durante dicho periodo temporal.

\subsection{Sprint 6 (Figura \ref{fig:sprint6})}

La gráfica de este sprint (Figura \ref{fig:sprint6}) muestra que se realizó una gran cantidad de trabajo al principio del sprint llegando la curva de trabajo real a cortar a la curva ideal un poco antes de la mitad de la misma. No obstante, al final del sprint hubo algún día de parón y el ritmo de trabajo se redujo. En consecuencia, no se alcanzaron los objetivos preestablecidos.

\subsection{Sprint 7 (Figura )}
\subsection{Sprint 8 (Figura )}
\subsection{Sprint 9 (Figura )}
\subsection{Sprint 10 (Figura )}

\section{Análisis global del proyecto}

\ctparttext{
  \color{black}
  \begin{center}
    Análisis de los resultados obtenidos.
  \end{center}
}
\part{Resultados obtenidos}

\chapter{Análisis de las correlaciones entre las distintas métricas}\label{chapter:correlations}
\addcontentsline{toc}{chapter}{Análisis de las correlaciones entre las distintas métricas}

En primer, las correlaciones que más nos interesan son las de las distintas métricas con la variable \emph{Grade}. En la Figura \ref{fig:correlations} podemos ver que no todas las métricas correlan con la misma.

\begin{figure}[H]
    \centering
    \includegraphics[width=\textwidth]{correlations.png}
    \caption{Correlaciones existentes entre las distintas métricas y la variable \emph{Grade}.}
    \label{fig:correlations}
\end{figure}

Así pues, vemos que las variables \emph{np} ($p = 0.0127 < 0.05$), \emph{fr} ($p = 0.0378 < 0.05$),    \emph{ps} ($p = 0.0031 < 0.05$) y \emph{sq} ($p = 0.0226 < 0.05$) correlan con la calificación obtenida y que la variable \emph{ns} podría correlar con la variable \emph{Grade} aunque con un grado de certeza menor que el resto ($p = 0.0712 < 0.1$).

También estudiaremos las correlaciones entre las medidas basadas en el análisis espectral de grafos y la variable \emph{Grade}. Empezaremos estudiando la correlación entre la medida \emph{LOGLAP09} anteriormente presentada y la calificación obtenida por los distintos grupos de prácticas.

\begin{figure}[H]
    \centering
    \includegraphics[width=0.60\textwidth]{correlaciones/outliersLOGLAP09.png}
    \caption{Distribución de los coeficientes \emph{LOGLAP09} obtenidos por los distintos grupos de alumnos inicial.}
    \label{fig:outliersLOGLAP09}
\end{figure}

En primer lugar, observamos la presencia de outliers en la distribución de la msima (Figura \ref{fig:outliersLOGLAP09}). Tras la eliminación de los mismos, podemos ver en la Figura  \ref{fig:correlationLOGLAP09} que la medida de rendimiento espectral \emph{LOGLAP09} no correla con la variable \emph{Grade} ($p = 0.5698 > 0.05$).

\begin{figure}[H]
    \centering
    \includegraphics[width=0.70\textwidth]{correlaciones/correlationLOGLAP09.png}
    \caption{Correlación existente entre la métrica \emph{LOGLAP09} y la variable \emph{Grade}.}
    \label{fig:correlationLOGLAP09}
\end{figure}
\chapter{Perfiles de estudiantes según su rendimiento}\label{sec:chapterXII}
\addcontentsline{toc}{chapter}{Perfiles de estudiantes según su rendimiento}

Predecir de manera exacta la calificación obtenida por los alumnos a partir de las medidas de rendimiento descritas anteriormente es imposible. No obstante, se agruparán las notas en clusters significativos y trataremos de predecir en qué cluster se encuentra la nota de un determinado grupo de alumnos. Finalmente, se conseguirá predecir las notas de los diferentes grupos de prácticas en intervalos con una variabilidad de $\pm 0.5$ puntos (Capítulo \ref{sec:chapterXIII}).

\section{Por clusters fijos de notas}

En primer lugar, escogeremos como separación los cuartiles de las calificaciones. Así pues, podemos ver la distribución de los cuartiles en la Figura \ref{fig:boxplotquartilegrade}, donde los límites inferiores de cada una de las cajas son $6.99$, $8.23$, $8.95$ y $9.60$ respectivamente. También puede verse en la Figura \ref{fig:countquartilegrade} el número de grupos que hay en cada cuartil.


\begin{figure}[H]
\centering
\subfloat[Boxplot de las calificaciones por cuartil.]{\label{fig:boxplotquartilegrade}\includegraphics[width=0.47\textwidth]{clustering/boxplotgrade.png}}\qquad
\subfloat[Número de grupos por cuartil.]{\label{fig:countquartilegrade}\includegraphics[width=0.47\textwidth]{clustering/countquartilegrade.png}}%
\caption{Resultados obtenidos tras agrupar las caficaciones por cuartiles.}
\label{fig:quartilegradeclustering}
\end{figure}

Sin embargo, en la Figura \ref{fig:boxplotquartilegrade} y en la Figura \ref{fig:frequenciesgrade}, donde se representan cómo de frecuentes son cada una de las calificaciones obtenidas, notamos la presencia de un outlier.

\begin{figure}[H]
    \centering
    \includegraphics[width=0.6\textwidth]{clustering/frequencygrade.png}
    \caption{Calificaciones obtenidas por los distintos grupos. El límite de los cuartiles se ha indicado con líneas verticales negras.}
    \label{fig:frequenciesgrade}
\end{figure}

%En la Figura \ref{fig:densitybyfactorquartilegrade} vemos las funciones de densidad por cuartil.
Prestaremos especial atención a los grupos del cluster \texttt{Q1} (el de las peores calificaciones al que le añadimos el outlier) puesto que son los que peor rendimiento han mostrado.

%\begin{figure}[H]
%    \centering
%    \includegraphics[width=0.6\textwidth]{clustering/densitybyfactorquartilegrade.png}
%    \caption{Funciones de densidad de las calificaciones obtenidas por cluster.}
%    \label{fig:densitybyfactorquartilegrade}
%\end{figure}

\section{Por clusters dinámicos de notas}

Se agruparán los datos usando el algoritmo de las K-medias sobre la variable \emph{Grade}. Para decidir el número de clusters en el que agruparemos los datos, se usarán métodos gráficos. Como podemos ver en las Figuras \ref{fig:indiceshubert} y \ref{fig:indicesdindex}, el número óptimo de particiones podría ser $3$ o $5$. Para decidir entre un número de clusters u otro se realizarán los dos agrupamientos y nos quedaremos con el de menor error.

\begin{figure}[H]
    \centering
    \includegraphics[width=\textwidth]{clustering/Hubert.png}
    \caption{Valores estadísticos de Hubert.}
    \label{fig:indiceshubert}
\end{figure}

\begin{figure}[H]
    \centering
    \includegraphics[width=\textwidth]{clustering/Dindex.png}
    \caption{Valores de Dindex.}
    \label{fig:indicesdindex}
\end{figure}

Aplicando el algoritmo de las K-medias para $K = 2$ (Figura \ref{fig:KMeans2}) y para $K = 3$ (Figura \ref{fig:KMeans3}), vemos que hay una gran diferencia entre la precisión de uno y otro (para $K = 2$ se tiene $\texttt{accuracy} = 0.6978412$ mientras que para $K = 3$ se tendrá $\texttt{accuracy} = 0.8585982$). Además, como puede apreciarse en las Figuras \ref{fig:KMeans2} y \ref{fig:KMeans3}, seguimos teniendo un outlier en ambos casos.

\begin{figure}[H]
    \centering
    \includegraphics[width=0.6\textwidth]{clustering/KMeans2.png}
    \caption{Particiones obtenidas con $K = 2$.}
    \label{fig:KMeans2}
\end{figure}

\begin{figure}[H]
    \centering
    \includegraphics[width=0.6\textwidth]{clustering/KMeans3.png}
    \caption{Particiones obtenidas con $K = 3$.}
    \label{fig:KMeans3}
\end{figure}

La distribución de la variable \emph{Grade} dentro de cada partición puede verse en las Figuras \ref{fig:KMeans2boxplot} y \ref{fig:KMeans3boxplot} mientras que el número de grupos que hay en las particiones puede verse en las Figuras \ref{fig:KMeans2count} y \ref{fig:KMeans3count}.

\begin{figure}[H]
\centering
\subfloat[Boxplot de cada una de las particiones.]{\label{fig:KMeans2boxplot}\includegraphics[width=0.47\textwidth]{clustering/KMeans2boxplot.png}}\qquad
\subfloat[Número de grupos por partición.]{\label{fig:KMeans2count}\includegraphics[width=0.47\textwidth]{clustering/KMeans2count.png}}%
\caption{Resultados obtenidos tras aplicar el algoritmo de las $K$-Medias con $K = 2$.}
\label{fig:KMeans2details}
\end{figure}

\begin{figure}[H]
\centering
\subfloat[Boxplot de cada una de las particiones.]{\label{fig:KMeans3boxplot}\includegraphics[width=0.47\textwidth]{clustering/KMeans3boxplot.png}}\qquad
\subfloat[Número de grupos por partición.]{\label{fig:KMeans3count}\includegraphics[width=0.47\textwidth]{clustering/KMeans3count.png}}%
\caption{Resultados obtenidos tras aplicar el algoritmo de las $K$-Medias con $K = 3$.}
\label{fig:KMeans3details}
\end{figure}

Por último, para cinco particiones (Figura \ref{fig:KMeans5}) se tendrá $\texttt{accuracy} = 0.9474439$. Es decir, tenemos más precisión con cinco particiones y ya no tenemos outliers. Nos centramos en estudiar los grupos de los dos primeros clusters (aquellos grupos con una nota inferior a $7.34$).

\begin{figure}[H]
    \centering
    \includegraphics[width=0.6\textwidth]{clustering/KMeans5.png}
    \caption{Particiones obtenidas con $K = 5$.}
    \label{fig:KMeans5}
\end{figure}

\begin{figure}[H]
\centering
\subfloat[Boxplot de cada una de las particiones.]{\label{fig:KMeans5boxplot}\includegraphics[width=0.47\textwidth]{clustering/KMeans5boxplot.png}}\qquad
\subfloat[Número de grupos por partición.]{\label{fig:KMeans5count}\includegraphics[width=0.47\textwidth]{clustering/KMeans5count.png}}%
\caption{Resultados obtenidos tras aplicar el algoritmo de las $K$-Medias con $K = 5$.}
\label{fig:KMeans5details}
\end{figure}

\section{Por clusters aproximados de rendimiento}

De las medidas de rendimiento estudiadas en el Capítulo \ref{chapter:rendimiento}, nos quedaremos con aquellas que correlan con la variable \emph{Grade} (\emph{np}, \emph{fr}, \emph{ps}, \emph{sq} y \emph{ns}). Así pues, se definirá una nueva métrica, a la que denotaremos por \emph{fm}, como la suma de las medidas de rendimiento \emph{p}, \emph{fr}, \emph{ps}, \emph{sq} y \emph{s}. En la Figura \ref{fig:correlationfm} vemos que \emph{fm} no correla con la variable \emph{Grade}.

\begin{figure}[H]
    \centering
    \includegraphics[width=0.8\textwidth]{clustering/fm.png}
    \caption{Regresión lineal para aproximar la relación de dependencia entre la variable \emph{fm} y la variable \emph{Grade}.}
    \label{fig:correlationfm}
\end{figure}

A continuación, se agruparán los datos usando el algoritmo de las K-medias sobre la variable \emph{fm}. Para decidir el número de clusters en el que agruparemos los datos, se usarán métodos gráficos. Como podemos ver en las Figuras \ref{fig:indiceshubertfm} y \ref{fig:indicesdindexfm}, el número óptimo de particiones podría ser $4$ o $7$. Para decidir entre un número de clusters u otro se realizarán los dos agrupamientos y nos quedaremos con el de menor error.

\begin{figure}[H]
    \centering
    \includegraphics[width=\textwidth]{clustering/Hubertfm.png}
    \caption{Valores estadísticos de Hubert.}
    \label{fig:indiceshubertfm}
\end{figure}

\begin{figure}[H]
    \centering
    \includegraphics[width=\textwidth]{clustering/Dindexfm.png}
    \caption{Valores de Dindex.}
    \label{fig:indicesdindexfm}
\end{figure}

Aplicando el algoritmo de las K-medias para $K = 4$ (Figura \ref{fig:KMeans4}) y para $K = 7$ (Figura \ref{fig:KMeans7}), vemos que hay una gran diferencia la precisión de uno y otro (para $K = 4$ se tiene $\texttt{accuracy} = 0.921137$ mientras que para $K = 7$ se tendrá $\texttt{accuracy} = 0.9772655$). Además, en la Figura \ref{fig:KMeans4} podemos notar la presencia de outliers mientras que en la Figura \ref{fig:KMeans7} no.

\begin{figure}[H]
    \centering
    \includegraphics[width=0.6\textwidth]{clustering/outliersfm.png}
    \caption{Particiones obtenidas con $K = 4$. Como podemos ver, se observa la presencia de outliers ($260.804$, $63.173$, $106.379$ y $261.823$).}
    \label{fig:KMeans4}
\end{figure}

\begin{figure}[H]
    \centering
    \includegraphics[width=0.6\textwidth]{clustering/KMeans7fm.png}
    \caption{Particiones obtenidas con $K = 7$. No hay ningún outlier.}
    \label{fig:KMeans7}
\end{figure}

La distribución de la variable \emph{Grade} dentro de cada partición puede verse en las Figuras \ref{fig:KMeans4boxplot} y \ref{fig:KMeans7boxplot} mientras que el número de grupos que hay en las particiones puede verse en las Figuras \ref{fig:KMeans4count} y \ref{fig:KMeans7count}.

\begin{figure}[H]
\centering
\subfloat[Boxplot de cada una de las particiones.]{\label{fig:KMeans4boxplot}\includegraphics[width=0.47\textwidth]{clustering/KMeansfmboxplot.png}}\qquad
\subfloat[Número de grupos por partición.]{\label{fig:KMeans4count}\includegraphics[width=0.47\textwidth]{clustering/KMeansfmcount.png}}%
\caption{Resultados obtenidos tras aplicar el algoritmo de las $K$-Medias con $K = 4$.}
\label{fig:KMeans4details}
\end{figure}

\begin{figure}[H]
\centering
\subfloat[Boxplot de cada una de las particiones.]{\label{fig:KMeans7boxplot}\includegraphics[width=0.47\textwidth]{clustering/KMeans7boxplot.png}}\qquad
\subfloat[Número de grupos por partición.]{\label{fig:KMeans7count}\includegraphics[width=0.47\textwidth]{clustering/KMeans7count.png}}%
\caption{Resultados obtenidos tras aplicar el algoritmo de las $K$-Medias con $K = 7$.}
\label{fig:KMeans7details}
\end{figure}

\section{Clustering mediante las propiedades espectrales de los grafos}
\subsection{Clustering mediante el coeficiente St}

Ahora, se ha decidido asociar los datos usando el algoritmo de las K-medias sobre la variable \emph{St}. Para fijar el número de clusters en el que agruparemos los datos, se usarán métodos gráficos. Como podemos ver en las Figuras \ref{fig:indiceshubertLAP} y \ref{fig:indicesdindexLAP}, se ha decidido agrupar los datos en $5$ clusters (Figura \ref{fig:KMeansLAP}, $\texttt{accuracy} = 0.9544702$).

\begin{figure}[H]
    \centering
    \includegraphics[width=\textwidth]{clustering/HubertLAP.png}
    \caption{Valores estadísticos de Hubert.}
    \label{fig:indiceshubertLAP}
\end{figure}

\begin{figure}[H]
    \centering
    \includegraphics[width=\textwidth]{clustering/DindexLAP.png}
    \caption{Valores de Dindex.}
    \label{fig:indicesdindexLAP}
\end{figure}

\begin{figure}[H]
    \centering
    \includegraphics[width=0.6\textwidth]{clustering/partitionsLOGLAP09.png}
    \caption{Particiones obtenidas con $K = 5$. No hay ningún outlier.}
    \label{fig:KMeansLAP}
\end{figure}

La distribución de la variable \emph{Grade} dentro de cada partición puede verse en la Figura \ref{fig:KMeansLAPboxplot} mientras que el número de grupos que hay en las particiones puede verse en la Figura \ref{fig:KMeansLAPcount}.

\begin{figure}[H]
\centering
\subfloat[Boxplot de cada una de las particiones.]{\label{fig:KMeansLAPboxplot}\includegraphics[width=0.47\textwidth]{clustering/KMeansLAPboxplot.png}}\qquad
\subfloat[Número de grupos por partición.]{\label{fig:KMeansLAPcount}\includegraphics[width=0.47\textwidth]{clustering/KMeansLAPcount.png}}%
\caption{Resultados obtenidos tras aplicar el algoritmo de las $K$-Medias con $K = 5$.}
\label{fig:KMeansLAPdetails}
\end{figure}

\subsection{Clustering mediante el coeficiente Ba}

Por último, se agruparán los datos usando el algoritmo de las K-medias sobre la variable \emph{Ba}. Para fijar el número de clusters en el que agruparemos los datos, se usarán métodos gráficos. Como podemos ver en las Figuras \ref{fig:indiceshubertBa} y \ref{fig:indicesdindexBa}, se ha decidido agrupar los datos en $5$ clusters (Figura \ref{fig:KMeansBa}, $\texttt{accuracy} = 0.9421558$).

\begin{figure}[H]
    \centering
    \includegraphics[width=\textwidth]{clustering/HubertBa.png}
    \caption{Valores estadísticos de Hubert.}
    \label{fig:indiceshubertBa}
\end{figure}

\begin{figure}[H]
    \centering
    \includegraphics[width=\textwidth]{clustering/DindexBa.png}
    \caption{Valores de Dindex.}
    \label{fig:indicesdindexBa}
\end{figure}

\begin{figure}[H]
    \centering
    \includegraphics[width=0.6\textwidth]{clustering/partitionsBa.png}
    \caption{Particiones obtenidas con $K = 5$. No hay ningún outlier.}
    \label{fig:KMeansBa}
\end{figure}

La distribución de la variable \emph{Grade} dentro de cada partición puede verse en la Figura \ref{fig:KMeansBaboxplot} mientras que el número de grupos que hay en las particiones puede verse en la Figura \ref{fig:KMeansBacount}.

\begin{figure}[H]
\centering
\subfloat[Boxplot de cada una de las particiones.]{\label{fig:KMeansBaboxplot}\includegraphics[width=0.47\textwidth]{clustering/KMeansBaboxplot.png}}\qquad
\subfloat[Número de grupos por partición.]{\label{fig:KMeansBacount}\includegraphics[width=0.47\textwidth]{clustering/KMeansBacount.png}}%
\caption{Resultados obtenidos tras aplicar el algoritmo de las $K$-Medias con $K = 5$.}
\label{fig:KMeansLAPdetails}
\end{figure}

\section{¿Quiénes son los grupos en riesgo?}\label{sec:badstudents}

El objetivo principal de este enfoque es tratar de identificar a los estudiantes que tienen dificultades en resolver las tareas propuestas en el laboratorio virtual porque ellos, y no los alumnos ordinarios, requieren en mayor medida de la atención del profesorado para poder alcanzar un $100\%$ de éxito. Pero, ¿quiénes son?

En principio nos vamos a centrar en los alumnos con el cuartil más bajo de la nota, es decir, aquellos cuya nota está por debajo, como mínimo, del $75\%$ de las demás calificaciones (Figura \ref{fig:frequenciesgrade}). Sin embargo, dada la dinámica de las calificaciones cada año (Figura \ref{fig:boxplotachieveryear}), podemos ver que este primer cuartil podría no ser muy preciso. En su lugar, se ha usado la función \texttt{KMeans} en R para dividir el rango de calificaciones en $5$ intervalos con una precisión del $95\%$ en las interdependencias e intradependencias de los $5$ clusters. En la partición obtenida mediante KMeans, hay dos clusters cuyos grupos han obtenido calificaciones estrictamente inferiores a $8.1$ (Figura \ref{fig:KMeans5}). Vamos a centrarnos en dichos grupos, a los que se denominará \emph{``LOW''}. Por el contrario, el resto de grupos se llamarán \emph{``GOOD''}.

Este conjunto de datos podría procesarse como un único cluster de datos y sólo tendríamos una visión del final del periodo de las prácticas, que podría ser informativo, pero no es útil para una intervención temprana. Con el fin de preparar el conjunto de datos para que sea útil cuanto antes y no sólo al final de la práctica, se va a estratificar por niveles donde cada nivel $i \in [3,9]$ está marcado por la primera vez que el problema $i$ se ha resuelto (Figura \ref{fig:stratification}). Por lo tanto, alcanzar nivel $9$ significa que se han completado todas las tareas.

\begin{figure}[H]
    \centering
    \includegraphics[width=\textwidth]{clustering/stratification.png}
    \caption{Estratificación del conjunto de datos en $10$ episodios consecutivos. Uno para la consecución de cada uno de los $9$ problemas, más un último episodio hasta el final de la práctica.}
    \label{fig:stratification}
\end{figure}

Así pues, tenemos un conjunto de datos muy fiable, reforzado con evidencias de que no hay sesgos a lo largo de los años. También tenemos una partición preliminar sólida y significativa de las calificaciones y un nuevo conjunto de funciones en forma de campana de uso general para identificar estas particiones (Sección \ref{sec:complexity}).
\chapter{Clasificación de los grupos de alumnos según su rendimiento}
\addcontentsline{toc}{chapter}{Clasificación de los grupos de alumnos según su rendimiento}


\begin{appendices}
\chapter{Algunas tablas}
\begin{table}[H]
\centering
\caption{Listado de los grupos por curso académico.}
\label{tab:groups1}
\begin{tabular}{cccc}
\hline
\textbf{Y2015} & \textbf{Y2016} & \textbf{Y2017} & \textbf{Y2018} \\ \hline
DBA 1516 P2 GA & DBA 1617 P2 GA & DBA 1718 P2 GA & DBA 1819 P2 GB \\
DBA 1516 P2 GB & DBA 1617 P2 GB & DBA 1718 P2 GB & DBA 1819 P2 GC \\
DBA 1516 P2 GC & DBA 1617 P2 GD & DBA 1718 P2 GC & DBA 1819 P2 GD \\
DBA 1516 P2 GD & DBA 1617 P2 GE & DBA 1718 P2 GD & DBA 1819 P2 GE \\
DBA 1516 P2 GE & DBA 1617 P2 GF & DBA 1718 P2 GE & DBA 1819 P2 GF \\
DBA 1516 P2 GF & DBA 1617 P2 GG & DBA 1718 P2 GG & DBA 1819 P2 GG \\
DBA 1516 P2 GG & DBA 1617 P2 GH & DBA 1718 P2 GH & DBA 1819 P2 GH \\
DBA 1516 P2 GH & DBA 1617 P2 GI & 				 & DBA 1819 P2 GI \\
DBA 1516 P2 GI & DBA 1617 P2 GJ & 				 & DBA 1819 P2 GJ \\
			   &				&			     & DBA 1819 P2 GK \\
			   &				&				 & DBA 1819 P2 GL \\ \hline
\end{tabular}
\end{table}

\begin{table}[H]
\centering
\caption{Listado de los grupos por curso académico.}
\label{tab:groups2}
\begin{tabular}{ccc}
\hline
\textbf{Y2019} & \textbf{Y2020} & \textbf{Y2021} \\ \hline
DBA 1920 P2 GB & DBA 2021 P2 GA & DBA 2122 P2 GA \\
DBA 1920 P2 GC & DBA 2021 P2 GB & DBA 2122 P2 GB \\
DBA 1920 P2 GD & DBA 2021 P2 GC & DBA 2122 P2 GC \\
DBA 1920 P2 GE & DBA 2021 P2 GD & DBA 2122 P2 GD \\
DBA 1920 P2 GF & DBA 2021 P2 GE & DBA 2122 P2 GE \\
DBA 1920 P2 GH & DBA 2021 P2 GF & DBA 2122 P2 GF \\
DBA 1920 P2 GI & DBA 2021 P2 GG & DBA 2122 P2 GG \\
DBA 1920 P2 GJ & DBA 2021 P2 GH & DBA 2122 P2 GH \\
DBA 1920 P2 GK & DBA 2021 P2 GI & DBA 2122 P2 GI \\
DBA 1920 P2 GL & DBA 2021 P2 GJ & DBA 2122 P2 GJ \\
DBA 1920 P2 GM & DBA 2021 P2 GK & DBA 2122 P2 GK \\
DBA 1920 P2 GN & DBA 2021 P2 GL & DBA 2122 P2 GL \\
			   & DBA 2021 P2 GM & DBA 2122 P2 GM \\
			   &         		& DBA 2122 P2 GN \\
			   & 				& DBA 2122 P2 GO \\
			   &				& DBA 2122 P2 GP \\ \hline
\end{tabular}
\end{table}

\begin{longtable}{llrrr}
\hline
\textbf{Year} & \textbf{Group} &  \textbf{fail} &  \textbf{solved} &  \textbf{all} \\
\hline
Y2015 & DBA 1516 P2 GA &   738 &      54 &  792 \\
Y2015 & DBA 1516 P2 GB &    62 &      49 &  111 \\
Y2015 & DBA 1516 P2 GC &   142 &     195 &  337 \\
Y2015 & DBA 1516 P2 GD &   298 &      80 &  378 \\
Y2015 & DBA 1516 P2 GE &   597 &     139 &  736 \\
Y2015 & DBA 1516 P2 GF &   246 &     110 &  356 \\
Y2015 & DBA 1516 P2 GG &   398 &      64 &  462 \\
Y2015 & DBA 1516 P2 GH &   525 &     181 &  706 \\
Y2015 & DBA 1516 P2 GI &   469 &     142 &  611 \\
Y2016 & DBA 1617 P2 GA &   132 &      59 &  191 \\
Y2016 & DBA 1617 P2 GB &   564 &     178 &  742 \\
Y2016 & DBA 1617 P2 GD &   154 &     208 &  362 \\
Y2016 & DBA 1617 P2 GE &   258 &     316 &  574 \\
Y2016 & DBA 1617 P2 GF &   126 &      47 &  173 \\
Y2016 & DBA 1617 P2 GG &   680 &     187 &  867 \\
Y2016 & DBA 1617 P2 GH &   722 &     161 &  883 \\
Y2016 & DBA 1617 P2 GI &   252 &     122 &  374 \\
Y2016 & DBA 1617 P2 GJ &   333 &      39 &  372 \\
Y2017 & DBA 1718 P2 GA &   186 &      46 &  232 \\
Y2017 & DBA 1718 P2 GB &  1282 &     139 & 1421 \\
Y2017 & DBA 1718 P2 GC &   369 &      73 &  442 \\
Y2017 & DBA 1718 P2 GD &   369 &      74 &  443 \\
Y2017 & DBA 1718 P2 GE &   468 &      48 &  516 \\
Y2017 & DBA 1718 P2 GG &   156 &     178 &  334 \\
Y2017 & DBA 1718 P2 GH &   235 &      38 &  273 \\
Y2018 & DBA 1819 P2 GB &   148 &       0 &  148 \\
Y2018 & DBA 1819 P2 GC &   178 &       0 &  178 \\
Y2018 & DBA 1819 P2 GD &   190 &       0 &  190 \\
Y2018 & DBA 1819 P2 GE &   158 &       0 &  158 \\
Y2018 & DBA 1819 P2 GF &   190 &       0 &  190 \\
Y2018 & DBA 1819 P2 GG &   266 &       0 &  266 \\
Y2018 & DBA 1819 P2 GH &   434 &       0 &  434 \\
Y2018 & DBA 1819 P2 GI &   242 &       0 &  242 \\
Y2018 & DBA 1819 P2 GJ &   373 &       0 &  373 \\
Y2018 & DBA 1819 P2 GK &   575 &       0 &  575 \\
Y2018 & DBA 1819 P2 GL &    57 &       0 &   57 \\
Y2019 & DBA 1920 P2 GB &   179 &      71 &  250 \\
Y2019 & DBA 1920 P2 GC &   366 &     116 &  482 \\
Y2019 & DBA 1920 P2 GD &   238 &     110 &  348 \\
Y2019 & DBA 1920 P2 GE &   266 &      63 &  329 \\
Y2019 & DBA 1920 P2 GF &   840 &     271 & 1111 \\
Y2019 & DBA 1920 P2 GH &   206 &      54 &  260 \\
Y2019 & DBA 1920 P2 GI &   119 &      37 &  156 \\
Y2019 & DBA 1920 P2 GJ &   588 &      48 &  636 \\
Y2019 & DBA 1920 P2 GK &   599 &     222 &  821 \\
Y2019 & DBA 1920 P2 GL &   388 &      56 &  444 \\
Y2019 & DBA 1920 P2 GM &   124 &      46 &  170 \\
Y2019 & DBA 1920 P2 GN &   122 &      27 &  149 \\
Y2020 & DBA 2021 P2 GA &   265 &      99 &  364 \\
Y2020 & DBA 2021 P2 GB &   221 &     174 &  395 \\
Y2020 & DBA 2021 P2 GC &   189 &      88 &  277 \\
Y2020 & DBA 2021 P2 GD &   104 &     231 &  335 \\
Y2020 & DBA 2021 P2 GE &    30 &      23 &   53 \\
Y2020 & DBA 2021 P2 GF &   138 &      28 &  166 \\
Y2020 & DBA 2021 P2 GG &   142 &      99 &  241 \\
Y2020 & DBA 2021 P2 GH &   250 &      31 &  281 \\
Y2020 & DBA 2021 P2 GI &   205 &     136 &  341 \\
Y2020 & DBA 2021 P2 GJ &   376 &      82 &  458 \\
Y2020 & DBA 2021 P2 GK &   177 &     105 &  282 \\
Y2020 & DBA 2021 P2 GL &   517 &      93 &  610 \\
Y2020 & DBA 2021 P2 GM &    60 &      37 &   97 \\
Y2021 & DBA 2122 P2 GA &   336 &      48 &  384 \\
Y2021 & DBA 2122 P2 GB &   471 &      28 &  499 \\
Y2021 & DBA 2122 P2 GC &   516 &      39 &  555 \\
Y2021 & DBA 2122 P2 GD &   347 &      34 &  381 \\
Y2021 & DBA 2122 P2 GE &   168 &      23 &  191 \\
Y2021 & DBA 2122 P2 GF &   418 &      37 &  455 \\
Y2021 & DBA 2122 P2 GG &   331 &      37 &  368 \\
Y2021 & DBA 2122 P2 GH &   258 &      45 &  303 \\
Y2021 & DBA 2122 P2 GI &   490 &      59 &  549 \\
Y2021 & DBA 2122 P2 GJ &   273 &      30 &  303 \\
Y2021 & DBA 2122 P2 GK &   425 &      39 &  464 \\
Y2021 & DBA 2122 P2 GL &   232 &      19 &  251 \\
Y2021 & DBA 2122 P2 GM &   513 &      39 &  552 \\
Y2021 & DBA 2122 P2 GN &   169 &      18 &  187 \\
Y2021 & DBA 2122 P2 GO &   359 &      40 &  399 \\
Y2021 & DBA 2122 P2 GP &   262 &      10 &  272 \\
\hline
\caption{Número y tipo de las sesiones de trabajo.}
\label{tab:type}
\end{longtable}
\end{appendices}

% Añade sección de referencias al final del documento.
% Selecciona un estilo de cita.
\bibliographystyle{apa-good}

% Añade la bibliografía al índice
\phantomsection
\addcontentsline{top}{chapter}{Bibliografía}
\bibliography{research} 

\end{document}