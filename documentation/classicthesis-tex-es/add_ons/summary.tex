\chapter*{Summary}
%\addcontentsline{toc}{chapter}{Summary} 

\begin{otherlanguage}{english}

The study of the learning process of students when they are given a series of tasks is fundamental since it could facilitate the assimilation of new knowledge and make what is being taught accessible to the students.

The main objective of this work is, precisely, to identify the behavioural patterns of those students at risk of underachieving in order to allow early intervention by teaching staff, preventing their failure in the acquisition of new knowledge.

Thus, in this project, data obtained in a virtual laboratory will be used to extract, through Process Mining techniques, a series of graphs, represented by matrices, which reflect the behaviour of the students on the platform as it will be demonstrated in this study.

An important part of this thesis has been the development of my own Process Mining tool, \emph{Graph Miner}, which is responsible for translating the virtual lab logs into the matrices that represent the behaviour of the learners. As we will see, it will make up for the drawbacks of the existing Process Mining programme, \emph{Disco}. \cite{gunther2012disco}.

In addition, supervised machine learning techniques will be used not only to predict groups at risk but also to predict in which range of grades the different groups of students are with statistical evidence. In fact, such predictions can be made with a high reliability at early stages of the development of students' practical work.

Finally, it should be noted that, in order to carry out the classifications described in the previous paragraph, both classical measures of student performance and measures of complexity based solely on the topology of the graph represented by the characteristic matrix of each group have been used to make the rankings described in the previous paragraph, the latter proving to be just as useful as the former.

\small{\spacedallcaps{Keywords:} learning analytics, \; learning styles, \; process mining, \; student success, \; graph theory, \; directed acyclic graph}

\end{otherlanguage}