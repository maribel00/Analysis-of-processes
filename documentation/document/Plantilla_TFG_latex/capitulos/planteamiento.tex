El servidor contiene varios mundos virtuales. Cada mundo virtual es una matriz cuadrada que representa espacios abiertos (en color blanco), obstáculos (en negro) y objetivos (en rojo) tal y como se muestra en la Figura \ref{fig:1}. Los agentes de los alumnos deben entrar en uno de esos mundos virtuales, percibir su vecindario, navegar a través de los espacios abiertos (empleando alguna clase de heurística exploratoria), evitar obstáculos y tratar de llegar al objetivo.

La percepción del agente de su entorno es crítica para resolver estos mundos. En este laboratorio virtual los alumnos pueden configurar cuál de los siguientes sensores estarán enchufados en sus agentes (cualquier combinación de ellos):

\begin{itemize}
	\item Un \textbf{GPS} que indica al agente sus coordenadas $(x,y)$ en el mundo virtual.
	\item Un \textbf{sensor de batería}. Cada agente está alimentado con una batería cuya capacidad es limitada y cuya carga decrece conforme el agente realiza algún movimiento. La batería nunca debe ser vaciada por completo.
	\item Un \textbf{sensor radar} que informa al agente acerca de los tipos de celdas que lo rodean con una percepción local de 5x5 (observar Figura \ref{fig:2}).
	\item Un \textbf{sensor escáner} que actúa como \emph{detector del objetivo} e indica al agente la distancia al objetivo medida desde cada una las celdas de su entorno 5x5 (observar Figura \ref{fig:3}).
\end{itemize}

Basados en su percepción del mundo virtual, cada agente decidirá ejecutar alguna de las siguientes acciones en su entorno implementando cualquier heurística o proceso de búsqueda.

\begin{itemize}
	\item LOGIN. Entrar en cualquiera de los mundos virtuales.
	\item MOVE. Mover al agente a una de las $8$ celdas adyacentes y gastar una cierta cantidad de batería. Si la celda destino es un obstáculo o el agente se queda sin batería, el agente se rompe y se sale del mundo  virtual.
	\item REFUEL. El agente recarga completamente su batería. A los agentes se les permite recargar su batería tantas veces como deseen.
\end{itemize}