La necesidad de comprender el proceso de aprendizaje y de personalizar la enseñanza para realizar una mejor adaptación a las necesidades del individuo ha motivado la \emph{Analítica de Aprendizaje} o \emph{Learning Analytics}, disciplina que consiste en la recogida de datos de un entorno de aprendizaje y el análisis de los mismos cuyo objetivo es asistir en el proceso de aprendizaje del alumnado.

Además, el uso de laboratorios virtuales y remotos en la enseñanza está en auge. Entre muchas de sus ventajas tenemos una mayor privacidad para el alumnado, accesos planficados a los mismos o soporte para reportar la actividad de los alumnos y la calificación de los mismos.

En este trabajo fin de grado se usarán datos de cinco cursos académicos obtenidos en el labotorio virtual para sistemas multiagente de la asignatura del cuarto curso académico Desarrollo Basado en Agentes del grado de Ingeniería Informática de la Universidad de Granada (España).

El laboratorio virtual diseñado para la asignatura recoge el trabajo diario de los alumnos almacenando las interacción entre los diferentes agentes y obteniendo así un extenso dataset que nos proporciona una base sólida para el uso de diversas analíticas de aprendizaje.

Así pues, se empleará un enfoque \emph{``data-driven''} o \emph{impulsado por datos}, tomando decisiones estratégicas basándose en el análisis de los datos y en la interpretación de los mismos.