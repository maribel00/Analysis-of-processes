\documentclass[10pt,a4paper]{article}
\usepackage[utf8]{inputenc}
\usepackage{amsmath}
\usepackage{amsfonts}
\usepackage{amssymb}
\usepackage{graphicx}
\usepackage[hidelinks]{hyperref} 
\usepackage{color}
\usepackage{xcolor}
\usepackage{caption}
\usepackage{subcaption}
\author{María Isabel Ruiz Martínez}
\title{Herramientas}

%Ruta absoluta en formato tipo Unix (Linux, OsX)
\graphicspath{ {/home/maribel/Escritorio/5º DGIIM/TFG/Analysis-of-processes/documentation/images} }

\begin{document}

\maketitle

En la realización de un proyecto investigación, la elección de las herramientas para su desarrollo es clave. A continuación se expone una lista con las que se han utilizado este trabajo fin de grado:

\begin{itemize}
	\item En el desarrollo del software se han empleado los lenguajes de
programación \texttt{R} y \texttt{C++}.
	\item Se ha empleado la herramienta de míneria de procesos \texttt{DISCO}.
	\item Se ha utilizado \texttt{GitHub} para alojar todo el 			contenido del proyecto y gestionar las distintas versiones del mismo.
	\item Para la redacción de la memoria se ha empleado el editor gratuito \texttt{Texmaker} utilizando el sistema de composición de textos \LaTeX.
	\item Para el seguimiento del proyecto se han utilizado las \texttt{Hojas de cálculo de Google}.
\end{itemize}

\end{document}