\documentclass[10pt,a4paper]{article}
\usepackage[utf8]{inputenc}
\usepackage{amsmath}
\usepackage{amsfonts}
\usepackage{amssymb}
\usepackage{graphicx}
\usepackage[hidelinks]{hyperref} 
\usepackage{color}
\usepackage{xcolor}
\usepackage{caption}
\usepackage{subcaption}
\author{María Isabel Ruiz Martínez}
\title{Motivación}

%Ruta absoluta en formato tipo Unix (Linux, OsX)
\graphicspath{ {/home/maribel/Escritorio/5º DGIIM/TFG/Analysis-of-processes/documentation/images} }

\begin{document}

\maketitle

\begin{quote}
\small El buen maestro hace que el mal estudiante se convierta en bueno y el buen estudiante en superior.

\rightline{\small Maruja Torres}
\end{quote}

Como decía Maruja Torres, un buen profesor puede influenciar positivamente en el desarrollo académico de sus alumnos. Así pues, uno de las principales motivaciones de este trabajo es, precisamente, ayudar a identificar buenas y malas prácticas de los alumnos precozmente para que el profesorado pueda asistirles mejor durante su proceso de aprendizaje y mejorar su rendimiento académico.

Así pues, la motivación última de los estudios expuesto en este documento es ayudar a aprender al alumnado.

Asimismo, creo que es de gran interés práctico conocer cuáles son las mejores estrategias a la hora de programar un sistema multiagente que desconoce el entorno que le rodea y que intenta conseguir un objetivo preestablecido.

\end{document}