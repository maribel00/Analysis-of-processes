\documentclass[10pt,a4paper]{article}
\usepackage[utf8]{inputenc}
\usepackage{amsmath}
\usepackage{amsfonts}
\usepackage{amssymb}
\usepackage{graphicx}
\usepackage[hidelinks]{hyperref} 
\usepackage{color}
\usepackage{xcolor}
\usepackage{caption}
\usepackage{subcaption}
\author{María Isabel Ruiz Martínez}
\title{Motivación}

%Ruta absoluta en formato tipo Unix (Linux, OsX)
\graphicspath{ {/home/maribel/Escritorio/5º DGIIM/TFG/Analysis-of-processes/documentation/images} }

\begin{document}

\maketitle

La necesidad de comprender el proceso de aprendizaje y de personalizar la enseñanza para realizar una mejor adaptación a las necesidades del individuo ha motivado la \emph{Analítica de Aprendizaje} o \emph{Learning Analytics}, disciplina que consiste en la recogida de datos de un entorno de aprendizaje y el análisis de los mismos cuyo objetivo es asistir en el proceso de aprendizaje del alumnado.

\end{document}